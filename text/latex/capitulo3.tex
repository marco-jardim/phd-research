\begin{chapter}{Justificativa}\label{cap:justificativa}

O \textit{linkage} (vinculação de registros), conforme apresentado no Capítulo \ref{cap:introducao}, constitui etapa indispensável para a integração de dados entre os múltiplos Sistemas de Informação em Saúde do Brasil \cite{Camargo2000reclink, Christen2012book}. A ausência de um identificador unívoco que perpasse bases como o SIM, o Sinan e o SIH-SUS torna necessário o emprego de métodos probabilísticos baseados na comparação de variáveis de identificação pessoal \cite{Fellegi1969theory, Camargo2000reclink}, cujos resultados dependem diretamente da qualidade dos dados disponíveis e da adequação dos limiares de classificação adotados.

O \textit{linkage} probabilístico, fundamentado no modelo de Fellegi e Sunter \cite{Fellegi1969theory}, é amplamente empregado em estudos epidemiológicos brasileiros por meio de ferramentas como o OpenRecLink \cite{Camargo2000reclink} e estratégias de blocagem \cite{Coeli2002blocking}. Persistem, contudo, desafios na classificação dos pares situados na ``área cinza'' do comparador (cf.\ Seção \ref{sec:area-cinza}): essa faixa intermediária de escores demanda revisão manual (\textit{clerical review}), procedimento dispendioso, pouco escalável e sujeito à variabilidade inter-avaliadores \cite{Christen2012book}. A resolução automatizada da área cinza constitui, portanto, o nó crítico que motiva a presente investigação.

\section{Lacuna do conhecimento}\label{sec:lacuna-conhecimento}

Embora a aplicação de técnicas de aprendizado de máquina (\textit{machine learning}) ao \textit{linkage} venha sendo investigada em contextos internacionais \cite{Christen2012book, Binette2022entity, Vo2019ensemble}, a literatura brasileira sobre o tema é incipiente, restringindo-se predominantemente a abordagens determinísticas e probabilísticas tradicionais. No cenário internacional, estudos de simulação em larga escala demonstraram que a escolha do método de vinculação pode introduzir viés sistemático nas estimativas populacionais \cite{Schnell2023microsimulation}. Estudos nacionais que empreguem classificadores supervisionados como pós-processamento do \textit{linkage} probabilístico, com vistas a automatizar a recuperação de pares na área cinza e a identificação de falsos positivos, são escassos. Soma-se a isso a carência de investigações que avaliem sistematicamente o impacto de diferentes estratégias de balanceamento de classes e de ajustes nos pontos de corte do comparador sobre a acurácia do processo de vinculação, apontando para uma lacuna relevante no campo da produção de dados vinculados em saúde no Brasil.

A maioria dos estudos brasileiros emprega protocolos padronizados de \textit{linkage} probabilístico cujos limiares são definidos empiricamente, sem análise sistemática da sensibilidade dos resultados a variações nesses parâmetros \cite{Camargo2000reclink, Coeli2002blocking}. A revisão manual da área cinza, quando realizada, configura etapa artesanal e não reprodutível, comprometendo a comparabilidade entre estudos \cite{Christen2012book}. Essa limitação agrava-se em contextos de crises sanitárias, nos quais a produção de informação oportuna é requisito para a tomada de decisão.

Faz-se necessário, portanto, investigar abordagens que reduzam a dependência da revisão manual e ampliem a recuperação de pares verdadeiros na área cinza. Estudos recentes em outros contextos demonstraram que abordagens baseadas em \textit{ensemble} de classificadores \cite{Vo2019ensemble} e métodos híbridos que combinam técnicas probabilísticas com aprendizado supervisionado \cite{Jiao2021hybrid, Almadani2026linking} podem elevar substancialmente a acurácia da vinculação. Para fins de saúde pública, o valor dessas estratégias também se expressa na redução do custo de \textit{clerical review} \cite{Christen2012book} e na possibilidade de mensurar o impacto epidemiológico da recuperação de pares verdadeiros em regiões de maior incerteza do escore.

\section{Justificativas específicas}\label{sec:justificativas-especificas}

Algumas justificativas específicas fundamentam a relevância deste estudo:

\begin{enumerate}

\item \textbf{Subnotificação da tuberculose e desfechos desfavoráveis.} A tuberculose (TB), reconhecida como condição marcadora da qualidade do cuidado em saúde \cite{Oliveira2012uso}, permanece como problema de saúde pública de grande magnitude no Brasil, com taxas de cura abaixo do preconizado pela Organização Mundial da Saúde e proporção não negligenciável de desfechos desfavoráveis, incluindo óbito, abandono e resistência medicamentosa \cite{WHO2024tb, Brasil2024tb}. Estudos anteriores demonstraram que o \textit{linkage} entre as bases do SIM e do Sinan-TB permite a identificação de óbitos por tuberculose não notificados ao sistema de vigilância, evidenciando subnotificação significativa \cite{Sousa2011obitos, Pinheiro2012subnotificacao, Rocha2015causas}. A melhoria na acurácia desse \textit{linkage} tem potencial para ampliar a capacidade de detecção de casos e a qualificação da informação epidemiológica, com impacto direto sobre a análise de causas múltiplas de óbito em coortes de pacientes com TB \cite{Rocha2015causas}.

\item \textbf{Intenso desbalanceamento de classes no \textit{linkage} SIM--Sinan.} O \textit{linkage} entre bases de mortalidade e de agravos de notificação gera um volume de pares candidatos no qual os pares verdadeiros constituem fração extremamente reduzida, frequentemente inferior a 1\% do total de comparações \cite{He2009imbalanced}. Esse desbalanceamento representa nó crítico para classificadores supervisionados e requer estratégias específicas de tratamento, cuja efetividade comparativa no contexto do \textit{record linkage} em saúde não se encontra adequadamente documentada na literatura brasileira, demandando investigação aprofundada. Hassani e colaboradores \citeyearpar{Hassani2025oversampling} propuseram recentemente uma estratégia combinada de sobreamostragem e subamostragem especificamente desenhada para \textit{linkage} de grande escala, evidenciando que o tratamento adequado do desbalanceamento pode elevar substancialmente o desempenho dos classificadores.

\item \textbf{Necessidade de protocolos reprodutíveis e automatizados.} A produção de dados vinculados para fins de vigilância epidemiológica e de pesquisa em serviços de saúde demanda agilidade e reprodutibilidade, especialmente em contextos de crises sanitárias nas quais a informação oportuna é requisito para a tomada de decisão no cuidado em saúde \cite{Viacava2012avaliacao}. A automatização de etapas do processo de classificação, mediante algoritmos treinados e validados, pode contribuir para a construção de protocolos padronizados de \textit{linkage} que reduzam a variabilidade e ampliem a escalabilidade do método. Nessa direção, diretrizes metodológicas internacionais já recomendam a integração de técnicas de aprendizado de máquina a dados vinculados para a estimação de indicadores populacionais de saúde \cite{Haneef2022guidelines}.

\item \textbf{Potencial de generalização para outros pares de bases de dados.} Embora o presente estudo tome como caso aplicado o \textit{linkage} SIM--Sinan-TB, as abordagens metodológicas desenvolvidas, incluindo as estratégias de balanceamento, os ajustes nos pontos de corte e os modelos de classificação, possuem potencial de aplicação a outros cenários de vinculação de bases de saúde no Brasil, como SIH-SUS--Sinan, SIM--Sinasc, entre outros, ampliando o alcance das contribuições para a produção de indicadores de desempenho do sistema de saúde \cite{Christen2012book,Paixao2017linkageevaluation}.

\item \textbf{Experiência institucional acumulada.} O Laboratório de Linkage e Análise de Dados Populacionais do Instituto de Estudo em Saúde Coletiva (IESC) da Universidade Federal do Rio de Janeiro (UFRJ) possui experiência de mais de 20 anos no \textit{linkage} de bases de dados de saúde no Brasil \cite{Camargo2000reclink, Coeli2002blocking, Oliveira2016accuracy}. Essa trajetória institucional fornece base sólida para o desenvolvimento e a validação de novas abordagens metodológicas, na medida em que dispõe de bases de dados previamente relacionadas, protocolos consolidados e equipe multidisciplinar com conhecimento tanto da área de saúde quanto de ciência de dados, potencializando a produção de conhecimento novo e útil para o campo da saúde coletiva.

\end{enumerate}


\section{A tuberculose como condição marcadora}\label{sec:tb-marcadora}

A escolha da tuberculose (TB) como condição de estudo neste trabalho fundamenta-se no conceito de condições traçadoras (\textit{tracer conditions}), proposto por Kessner, Kalk e Singer \citeyearpar{Kessner1973tracer}, segundo o qual determinadas condições de saúde podem funcionar como reveladoras do desempenho do sistema assistencial, desde que sejam inequivocamente identificáveis, possuam prevalência suficiente, tenham história natural modificável pela intervenção e disponham de técnicas de manejo bem estabelecidas. A TB atende a todos esses requisitos: é doença de notificação compulsória, registrada em múltiplos SIS (Sinan, SIM, SIH-SUS, GAL, SITETB), cujo tratamento é padronizado e disponibilizado integralmente pelo SUS \cite{PNCT2019manual}. Essas propriedades permitem que o percurso do paciente com TB na rede de serviços seja rastreável por meio do \textit{linkage}, revelando atrasos no diagnóstico, irregularidade no tratamento, abandono, internações evitáveis e óbitos que poderiam ter sido prevenidos \cite{Sousa2011obitos, Oliveira2012uso}.

Estudos conduzidos pelo grupo de pesquisa do IESC/UFRJ demonstraram que o \textit{linkage} entre o SIM e o Sinan-TB identificou óbitos por TB não notificados ao sistema de vigilância, evidenciando subnotificação expressiva \cite{Pinheiro2012subnotificacao, Sousa2011obitos, Oliveira2012uso}. Investigações subsequentes qualificaram variáveis do Sinan-TB por meio de regras de \textit{scripting} aplicadas sobre dados vinculados \cite{Rocha2019scripting} e analisaram as causas múltiplas de morte em coortes de pacientes notificados \cite{Rocha2015causas}. A taxa de cura no Brasil permanece abaixo do preconizado pela Organização Mundial da Saúde, e os índices de abandono persistem elevados \cite{Brasil2024tb, WHO2024tb}, indicando que a TB continua a revelar fragilidades na organização do cuidado. A melhoria da acurácia do \textit{linkage} entre essas bases tem, portanto, implicações diretas para a avaliação da efetividade do programa de controle da tuberculose.


\section{Urgência em contextos de crises sanitárias}\label{sec:urgencia-crises}

A necessidade de protocolos automatizados e reprodutíveis de \textit{linkage} é acentuada em contextos de crises sanitárias. A pandemia de COVID-19 provocou sobrecarga nos serviços de saúde, com redução documentada no número de notificações de tuberculose, interrupção de tratamentos e aumento de desfechos desfavoráveis \cite{Ranzani2021covid, Maia2022covid_tb}. A queda na detecção de casos durante a pandemia não refletiu redução na incidência da doença, mas a retração do acesso a diagnóstico e a desarticulação de rotinas de vigilância \cite{Hallal2020covid}. Cenários semelhantes podem ocorrer em crises climáticas e epidêmicas futuras, reforçando a importância de dispor de métodos de \textit{linkage} que possam ser executados de forma ágil e padronizada, sem depender exclusivamente de revisão manual.


\section{Vinculação institucional}\label{sec:vinculacao-institucional}

O presente trabalho insere-se no programa de pós-graduação do Instituto de Estudos em Saúde Coletiva (IESC) da Universidade Federal do Rio de Janeiro (UFRJ), no âmbito da linha de pesquisa em Ciência de Dados aplicada à Saúde. O IESC abriga o Laboratório de Linkage e Análise de Dados Populacionais, que desenvolve, há mais de duas décadas, metodologias de vinculação de bases de dados para a vigilância epidemiológica e a avaliação de serviços de saúde \cite{Camargo2000reclink, Coeli2002blocking}. O estudo conta ainda com a colaboração da Secretaria Acadêmica de Saúde, que articula atividades de ensino, pesquisa e extensão voltadas à qualificação dos dados em saúde e ao fortalecimento da capacidade analítica dos sistemas de informação do SUS. Essa vinculação institucional assegura o acesso a bases de dados previamente vinculadas, protocolos consolidados e expertise multidisciplinar necessários para o desenvolvimento e a validação das abordagens propostas.
A partir da contextualização ora apresentada, o capítulo seguinte enuncia os objetivos geral e específicos que norteiam a investigação.

\end{chapter}
