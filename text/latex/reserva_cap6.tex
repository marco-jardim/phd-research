% =============================================================================
% RESERVA PARA CAPÍTULO 7 (DISCUSSÃO)
% Seções de pensamento sistêmico, episódios de cuidado e painéis BI,
% reescritas para conectar-se aos resultados empíricos da tese.
% =============================================================================


% -----------------------------------------------------------------------------
% §7.3 - Pensamento sistêmico e crises sanitárias
% Conectado aos resultados: recuperação de óbitos, zona cinzenta, COVID-19
% -----------------------------------------------------------------------------
\section[Pensamento sistêmico e crises sanitárias]{Pensamento sistêmico e o impacto de crises sanitárias}\label{sec:pensamento-sistemico}

Sistemas de saúde constituem sistemas complexos adaptativos, compostos por múltiplos agentes que interagem de forma não linear e produzem dinâmicas emergentes irredutíveis à análise isolada de seus componentes \cite{Plsek2001complexity}. O pensamento sistêmico propõe que a compreensão de fenômenos de saúde exige a consideração de interrelações entre elementos, de rotas de retroalimentação (\textit{feedback loops}) e de efeitos não intencionais de intervenções, em contraposição à lógica reducionista que isola variáveis e relações causais lineares \cite{Sterman2000business, Luke2012systems}.

Os resultados desta tese oferecem evidência concreta dessa interdependência. A recuperação de 24 óbitos adicionais pelo classificador RF+SMOTE, em relação ao limiar ingênuo $\geq 8$, ilustra como uma falha localizada na cadeia de informação (a classificação imprecisa de potenciais pares na zona cinzenta) propaga seus efeitos para indicadores de nível populacional: taxas de mortalidade subestimadas, encerramentos de fichas de notificação incorretos e, consequentemente, alocação de recursos baseada em informação incompleta \cite{Bartholomay2014improved, Lima2020tbquality}. Sob a ótica sistêmica, o pós-processamento por aprendizado de máquina atua como mecanismo de correção de um ponto de fragilidade na rota de retroalimentação entre o registro do óbito e a vigilância epidemiológica.

A aplicação do pensamento sistêmico à saúde pública tem ganhado relevância na análise de problemas que envolvem múltiplos determinantes sociais, ambientais e organizacionais \cite{DiezRoux2011complexity, Adam2012systems}. No campo das doenças infecciosas, essa perspectiva permite reconhecer que o desfecho do tratamento de um paciente com tuberculose não depende exclusivamente da eficácia do esquema terapêutico, mas de uma rede de fatores que inclui o acesso oportuno ao diagnóstico, a organização dos serviços de atenção primária, a disponibilidade de exames laboratoriais e a capacidade de articulação entre os pontos da rede \cite{Mendes2011redes}. A concentração de aproximadamente 47\% dos pares verdadeiros na zona cinzenta (escores 5 a 8) reflete, em parte, essa complexidade: registros com preenchimento incompleto, variações na grafia de nomes ou inconsistências em datas de nascimento são manifestações, no nível dos dados, de fragilidades organizacionais nos pontos de registro do sistema de saúde.

Crises sanitárias recentes amplificam essas fragilidades de forma documentada. A pandemia de COVID-19 provocou sobrecarga nos serviços hospitalares e de atenção primária no Brasil, com redução no número de notificações de tuberculose, interrupção de tratamentos e aumento de desfechos desfavoráveis \cite{Ranzani2021covid, Maia2022covid_tb}. A queda na detecção de casos de TB durante a pandemia não reflete necessariamente redução na incidência da doença, mas retração do acesso aos serviços de diagnóstico e desarticulação de rotinas de vigilância \cite{Hallal2020covid}. Nesse cenário, a degradação adicional da qualidade dos registros (campos incompletos, atrasos na digitação, acúmulo de fichas não encerradas) tende a aumentar a proporção de pares candidatos que caem na zona cinzenta, tornando o pós-processamento supervisionado ainda mais necessário. A robustez do classificador demonstrada pela análise de sensibilidade ao desbalanceamento (F$_1$-Score entre 0,880 e 0,918 sob nove estratégias) sugere que o \textit{framework} proposto pode manter desempenho estável mesmo em períodos de deterioração da qualidade dos dados, embora essa hipótese requeira validação em coortes pandêmicas.

A contribuição do \textit{framework} configurável, nesse contexto, transcende o ganho preditivo: ao tornar explícitos os compromissos entre precisão e sensibilidade em cada ponto operacional da fronteira de Pareto (828 configurações exploradas), o sistema permite que gestores ajustem a sensibilidade do monitoramento conforme o cenário epidemiológico vigente. Em períodos de crise, a operação em ponto de maior sensibilidade pode compensar parcialmente a perda de notificações, funcionando como mecanismo de alerta para desorganizações sistêmicas detectáveis pelo aumento de vínculos recuperados na zona cinzenta.


% -----------------------------------------------------------------------------
% §7.4 - Episódios de cuidado e itinerário terapêutico
% Conectado aos resultados: SHAP, NOMEMAE, zona cinzenta, framework
% -----------------------------------------------------------------------------
\section{Episódios de cuidado e itinerário terapêutico}\label{sec:episodios-itinerario}

O conceito de episódio de cuidado, introduzido por Hornbrook, Hurtado e Johnson \citeyearpar{Hornbrook1985episodes}, designa o conjunto articulado de serviços de saúde prestados a um indivíduo em relação a um problema clínico específico, ao longo de um período temporal definido. Diferentemente da análise de eventos isolados (uma internação, uma consulta, um exame), a abordagem por episódios reconhece que o cuidado em saúde constitui processo longitudinal, no qual as interações do paciente com diferentes pontos do sistema assistencial são interdependentes.

A reconstrução de episódios de cuidado a partir de dados administrativos depende, fundamentalmente, da capacidade de identificar registros referentes a um mesmo indivíduo em diferentes bases de informação. Na ausência de um identificador unívoco no SUS, essa tarefa exige o emprego de técnicas de \textit{linkage} que possibilitem vincular, por exemplo, a notificação de um caso de TB no Sinan à eventual declaração de óbito no SIM \cite{Coeli2021suboptimal}. A acurácia dessa vinculação determina a completude do episódio reconstruído: cada par verdadeiro não identificado representa uma lacuna no itinerário terapêutico do paciente.

Os achados de interpretabilidade desta tese lançam luz sobre a dinâmica dessa reconstrução. A análise SHAP (\textit{SHapley Additive exPlanations}) revelou que, na zona cinzenta do escore agregado, o nome da mãe (NOMEMAE) emerge como atributo dominante nas decisões do classificador (Tabela \ref{tab:shap-importance}, Figura \ref{fig:shap-summary}) \cite{Lundberg2017shap, Lundberg2020treeshap}. Essa predominância possui interpretação epidemiológica relevante: quando a evidência de nome próprio e data de nascimento se torna ambígua (por erros de digitação, homônimos ou variações de grafia), o nome da mãe funciona como âncora de identificação familiar que preserva a vinculação mesmo diante de inconsistências nos demais campos. Tal padrão sugere que o itinerário terapêutico do paciente com TB, quando reconstruído por \textit{linkage}, depende criticamente da qualidade de preenchimento de campos que, na prática assistencial, são frequentemente tratados como secundários.

Essa constatação dialoga com a literatura sobre itinerário terapêutico na perspectiva antropológica. O itinerário, conceito que designa o percurso empreendido pelo indivíduo na busca por cuidado, englobando serviços formais, estratégias informais e barreiras de acesso \cite{Cabral2008itinerario, Gerhardt2006itinerario}, pode ser operacionalizado como sequência temporal de eventos registrados em diferentes sistemas quando o \textit{linkage} é bem-sucedido. Os 24 óbitos adicionais recuperados pelo RF+SMOTE representam, cada um, a restauração de um elo entre a trajetória de cuidado (registrada no Sinan) e o desfecho final (registrado no SIM). Sem essa vinculação, o episódio de cuidado permanece incompleto: o sistema de vigilância registra uma notificação sem encerramento definitivo, e o óbito permanece dissociado de sua história clínica, comprometendo tanto o cálculo de indicadores quanto a avaliação da qualidade da assistência prestada \cite{Bartholomay2020drtb}.

A transparência proporcionada pela análise SHAP amplia a utilidade do \textit{framework} para a gestão da informação em saúde \cite{Markus2021role}. Ao identificar que o nome da mãe é o atributo decisivo na zona de incerteza, o sistema fornece aos gestores uma evidência objetiva para direcionar ações de melhoria da qualidade do preenchimento: investir na completude do campo de nome da mãe nos formulários de notificação e nas declarações de óbito pode reduzir a proporção de pares que caem na zona cinzenta, diminuindo a dependência do pós-processamento supervisionado e fortalecendo a capacidade de reconstrução automática dos episódios de cuidado.


% -----------------------------------------------------------------------------
% §7.5 - Painéis de monitoramento e inteligência de dados em saúde
% Conectado aos resultados: framework configurável, pipelines, Pareto
% -----------------------------------------------------------------------------
\section{Painéis de monitoramento e inteligência de dados em saúde}\label{sec:paineis-bi}

A crescente disponibilidade de dados em saúde tem motivado o desenvolvimento de painéis de monitoramento (\textit{dashboards}) e sistemas de inteligência de dados (\textit{Business Intelligence}, BI) voltados à gestão e à vigilância em saúde \cite{Kimball2013dw}. Essas ferramentas permitem a agregação, a visualização e a análise de indicadores em tempo oportuno, subsidiando a tomada de decisão em diferentes níveis do sistema. Entretanto, parte expressiva dessas iniciativas limita-se à apresentação descritiva de dados provenientes de bases isoladas, sem incorporar a integração de fontes necessária para a construção de indicadores de processo e resultado.

Os \textit{pipelines} configuráveis propostos nesta tese podem alimentar painéis de monitoramento com dados vinculados de maior qualidade, expandindo o repertório de indicadores disponíveis para a gestão. No \textit{pipeline} de vigilância (RF+SMOTE, limiar 0,5), a saída classificada forneceria, em fluxo operacional, a lista atualizada de óbitos vinculados a notificações de TB, permitindo a construção de indicadores longitudinais: proporção de óbitos entre casos notificados, tempo entre notificação e óbito, e taxa de encerramento oportuno. A operação desse \textit{pipeline} em ponto de alta sensibilidade (F$_1$-Score = 0,916 $\pm$ 0,026 na validação cruzada) asseguraria cobertura ampla dos eventos, condição necessária para que o painel reflita a realidade epidemiológica com mínima subestimação \cite{Bartholomay2014improved}.

Já no \textit{pipeline} de confirmação (combinação híbrida AND), as classificações de alta confiança integrariam módulos de investigação individual nos painéis, apresentando ao epidemiologista apenas os pares de alta confiança que dispensam revisão adicional. A fronteira de Pareto, com 828 configurações exploradas, oferece ao gestor um mapa de possibilidades operacionais que pode ser incorporado à interface do painel como ferramenta de ajuste dinâmico: em períodos de sobrecarga (por exemplo, durante uma crise sanitária), o operador poderia deslocar o ponto operacional em direção a maior sensibilidade; em períodos de rotina, retornaria ao ponto de maior precisão, reduzindo o volume de alertas \cite{Harron2017linkagequality}.

A interpretabilidade do classificador via SHAP agrega uma dimensão adicional aos painéis. A exibição das contribuições dos atributos para cada decisão de classificação (em particular, a predominância do nome da mãe na zona cinzenta) permitiria que o painel funcionasse não apenas como ferramenta de monitoramento epidemiológico, mas também como instrumento de gestão da qualidade da informação: regiões ou unidades de saúde com alta proporção de pares classificados na zona cinzenta poderiam ser priorizadas para ações de capacitação em preenchimento de registros \cite{Lundberg2020treeshap, Markus2021role}. Essa integração entre a saída do \textit{framework} de \textit{linkage} e painéis de BI representa, assim, a materialização do ciclo completo de inteligência epidemiológica: do registro no ponto de atenção, passando pela vinculação probabilística e classificação supervisionada, até a geração de indicadores acionáveis para a tomada de decisão em saúde pública.
