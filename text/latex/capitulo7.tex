\begin{chapter}{Discussão}
\label{cap:discussao}

Este capítulo discute os achados apresentados no Capítulo \ref{cap:resultados} à luz da literatura sobre \textit{linkage} probabilístico, aprendizado de máquina em saúde e vigilância da tuberculose no Brasil. A discussão está organizada em seis eixos: (i) o impacto epidemiológico da recuperação de óbitos na zona cinzenta, (ii) a racionalidade operacional dos dois \textit{pipelines} propostos, (iii) a leitura dos resultados sob a ótica do pensamento sistêmico, (iv) as implicações para a reconstrução de episódios de cuidado, (v) o potencial de alimentação de painéis de monitoramento e (vi) as limitações do estudo e possibilidades de generalização.

\section{Impacto epidemiológico e operacional}
\label{sec:cap6-impacto-epidemiologico}

A subestimação de óbitos por tuberculose constitui problema recorrente nos sistemas de vigilância brasileiros, decorrente tanto da subnotificação de casos quanto de falhas no encerramento oportuno das fichas de investigação \cite{Bartholomay2014improved, Santos2018factors}. Quando o \textit{linkage} entre o Sistema de Informação sobre Mortalidade (SIM) e o Sistema de Informação de Agravos de Notificação (Sinan) é utilizado para recuperar esses óbitos, a acurácia da etapa de classificação dos pares candidatos determina diretamente a magnitude do viés de mensuração remanescente \cite{Doidge2019linkageerror, Shaw2022biases}.

Os resultados obtidos neste estudo quantificam esse impacto com precisão operacional. A configuração RF+SMOTE (limiar 0,5) recuperou 24 óbitos adicionais em relação ao limiar ingênuo $\geq 8$ do escore agregado, o que representa incremento de 55,8\% na detecção de pares verdadeiros no conjunto de teste (Tabela \ref{tab:impacto-epidemiologico}). Esse ganho não é marginal: cada óbito não vinculado ao registro de notificação impede a correção do desfecho na ficha do Sinan, comprometendo o cálculo de indicadores como a taxa de mortalidade entre casos notificados e a proporção de encerramentos por óbito \cite{Lima2020tbquality, Oliveira2019early}.

\begin{table}[htbp]
\centering
\caption{Comparação de métodos: óbitos detectados, custo operacional e taxa corrigida}
\label{tab:impacto-epidemiologico}
\footnotesize
\begin{tabular}{@{}lrrrrrl@{}}
\toprule
Método & Det. & Perd. & Rev. & Prec. & \textit{Recall} & \% Verd. \\
\midrule
Limiar ingênuo $\geq$7 & 58 & 16 & 404 & 0.144 & 0.784 & 78.4\% \\
Limiar ingênuo $\geq$8 & 43 & 31 & 67 & 0.642 & 0.581 & 58.1\% \\
Limiar ingênuo $\geq$9 & 31 & 43 & 31 & 1.000 & 0.419 & 41.9\% \\
Regras $\geq$7 & 55 & 19 & 62 & 0.887 & 0.743 & 74.3\% \\
Regras $\geq$8 & 36 & 38 & 36 & 1.000 & 0.486 & 48.6\% \\
ML RF+SMOTE $\geq$0.5 & 67 & 7 & 70 & 0.957 & 0.905 & 90.5\% \\
ML RF+SMOTE $\geq$0.7 & 64 & 10 & 65 & 0.985 & 0.865 & 86.5\% \\
ML GB $\geq$0.5 & 60 & 14 & 64 & 0.938 & 0.811 & 81.1\% \\
Híb.-OR RF$\geq$0.7+R$\geq$8 & 64 & 10 & 65 & 0.985 & 0.865 & 86.5\% \\
Híb.-AND RF$\geq$0.5+R$\geq$7 & 55 & 19 & 55 & 1.000 & 0.743 & 74.3\% \\
\bottomrule
\multicolumn{7}{@{}l}{\footnotesize Det.=Detectados; Perd.=Perdidos; Rev.=Revisões; Prec.=Precisão; Verd.=Verdadeiros.}
\end{tabular}
\end{table}

O custo operacional reforça a vantagem do pós-processamento supervisionado. A razão de revisões por par verdadeiro recuperado situou-se em aproximadamente 1,0 para RF+SMOTE, contra 1,6 para o limiar ingênuo $\geq 8$, indicando que o modelo de aprendizado de máquina concentra a revisão manual em candidatos com maior probabilidade de serem pares verdadeiros. Essa eficiência é particularmente relevante em contextos municipais, onde a capacidade de revisão clerical é limitada e o custo de oportunidade de cada registro avaliado erroneamente é alto \cite{Coeli2021suboptimal}.

A distribuição dos óbitos recuperados por faixa de escore (Tabela \ref{tab:perfil-recuperados}) confirma que o ganho se concentra na zona cinzenta (escores 5 a 8), região que abriga aproximadamente 47\% dos pares verdadeiros do conjunto de teste. Limiares fixos aplicados ao escore agregado falham precisamente nessa faixa, onde a heterogeneidade de erros de digitação, abreviações e variações nominais torna a evidência de similaridade insuficiente para decisão automatizada. A capacidade do classificador de explorar padrões multivariados nos 29 subescores de similaridade permite discriminar pares que seriam indistinguíveis por um único ponto de corte, resultado consistente com achados de Paixão et al. \citeyearpar{Paixao2017linkageevaluation} sobre a superioridade de abordagens combinadas em bases administrativas brasileiras.

\begin{table}[htbp]
\centering
\caption{Perfil dos óbitos recuperados pelo ML (não encontrados pelo limiar $\geq 8$)}
\label{tab:perfil-recuperados}
\begin{tabular}{lcc}
\toprule
Característica & Recuperados pelo ML & Já encontrados \\
\midrule
N & 29 & 42 \\
Idade (média $\pm$ DP) & 50.2 $\pm$ 13.2 & 51.0 $\pm$ 18.7 \\
Sexo masculino (\%) & 0 (0.0\%) & 0 (0.0\%) \\
Nota final (média) & 6.92 & 9.95 \\
\quad Faixa 5--6 & 3 (10.3\%) & 0 (0.0\%) \\
\quad Faixa 6--7 & 11 (37.9\%) & 0 (0.0\%) \\
\quad Faixa 7--8 & 15 (51.7\%) & 0 (0.0\%) \\
\bottomrule
\end{tabular}
\end{table}

\begin{figure}[!ht]
\centering
%% Creator: Matplotlib, PGF backend
%%
%% To include the figure in your LaTeX document, write
%%   \input{<filename>.pgf}
%%
%% Make sure the required packages are loaded in your preamble
%%   \usepackage{pgf}
%%
%% Also ensure that all the required font packages are loaded; for instance,
%% the lmodern package is sometimes necessary when using math font.
%%   \usepackage{lmodern}
%%
%% Figures using additional raster images can only be included by \input if
%% they are in the same directory as the main LaTeX file. For loading figures
%% from other directories you can use the `import` package
%%   \usepackage{import}
%%
%% and then include the figures with
%%   \import{<path to file>}{<filename>.pgf}
%%
%% Matplotlib used the following preamble
%%   \def\mathdefault#1{#1}
%%   \everymath=\expandafter{\the\everymath\displaystyle}
%%   \IfFileExists{scrextend.sty}{
%%     \usepackage[fontsize=10.000000pt]{scrextend}
%%   }{
%%     \renewcommand{\normalsize}{\fontsize{10.000000}{12.000000}\selectfont}
%%     \normalsize
%%   }
%%   
%%   \makeatletter\@ifpackageloaded{underscore}{}{\usepackage[strings]{underscore}}\makeatother
%%
\begingroup%
\makeatletter%
\begin{pgfpicture}%
\pgfpathrectangle{\pgfpointorigin}{\pgfqpoint{4.800000in}{3.000000in}}%
\pgfusepath{use as bounding box, clip}%
\begin{pgfscope}%
\pgfsetbuttcap%
\pgfsetmiterjoin%
\definecolor{currentfill}{rgb}{1.000000,1.000000,1.000000}%
\pgfsetfillcolor{currentfill}%
\pgfsetlinewidth{0.000000pt}%
\definecolor{currentstroke}{rgb}{1.000000,1.000000,1.000000}%
\pgfsetstrokecolor{currentstroke}%
\pgfsetdash{}{0pt}%
\pgfpathmoveto{\pgfqpoint{0.000000in}{0.000000in}}%
\pgfpathlineto{\pgfqpoint{4.800000in}{0.000000in}}%
\pgfpathlineto{\pgfqpoint{4.800000in}{3.000000in}}%
\pgfpathlineto{\pgfqpoint{0.000000in}{3.000000in}}%
\pgfpathlineto{\pgfqpoint{0.000000in}{0.000000in}}%
\pgfpathclose%
\pgfusepath{fill}%
\end{pgfscope}%
\begin{pgfscope}%
\pgfsetbuttcap%
\pgfsetmiterjoin%
\definecolor{currentfill}{rgb}{1.000000,1.000000,1.000000}%
\pgfsetfillcolor{currentfill}%
\pgfsetlinewidth{0.000000pt}%
\definecolor{currentstroke}{rgb}{0.000000,0.000000,0.000000}%
\pgfsetstrokecolor{currentstroke}%
\pgfsetstrokeopacity{0.000000}%
\pgfsetdash{}{0pt}%
\pgfpathmoveto{\pgfqpoint{0.554706in}{0.502083in}}%
\pgfpathlineto{\pgfqpoint{4.650000in}{0.502083in}}%
\pgfpathlineto{\pgfqpoint{4.650000in}{2.850000in}}%
\pgfpathlineto{\pgfqpoint{0.554706in}{2.850000in}}%
\pgfpathlineto{\pgfqpoint{0.554706in}{0.502083in}}%
\pgfpathclose%
\pgfusepath{fill}%
\end{pgfscope}%
\begin{pgfscope}%
\pgfpathrectangle{\pgfqpoint{0.554706in}{0.502083in}}{\pgfqpoint{4.095294in}{2.347917in}}%
\pgfusepath{clip}%
\pgfsetbuttcap%
\pgfsetmiterjoin%
\definecolor{currentfill}{rgb}{0.172549,0.627451,0.172549}%
\pgfsetfillcolor{currentfill}%
\pgfsetlinewidth{0.401500pt}%
\definecolor{currentstroke}{rgb}{0.000000,0.000000,0.000000}%
\pgfsetstrokecolor{currentstroke}%
\pgfsetdash{}{0pt}%
\pgfpathmoveto{\pgfqpoint{0.740856in}{0.502083in}}%
\pgfpathlineto{\pgfqpoint{1.192128in}{0.502083in}}%
\pgfpathlineto{\pgfqpoint{1.192128in}{1.717475in}}%
\pgfpathlineto{\pgfqpoint{0.740856in}{1.717475in}}%
\pgfpathlineto{\pgfqpoint{0.740856in}{0.502083in}}%
\pgfpathclose%
\pgfusepath{stroke,fill}%
\end{pgfscope}%
\begin{pgfscope}%
\pgfpathrectangle{\pgfqpoint{0.554706in}{0.502083in}}{\pgfqpoint{4.095294in}{2.347917in}}%
\pgfusepath{clip}%
\pgfsetbuttcap%
\pgfsetmiterjoin%
\definecolor{currentfill}{rgb}{0.172549,0.627451,0.172549}%
\pgfsetfillcolor{currentfill}%
\pgfsetlinewidth{0.401500pt}%
\definecolor{currentstroke}{rgb}{0.000000,0.000000,0.000000}%
\pgfsetstrokecolor{currentstroke}%
\pgfsetdash{}{0pt}%
\pgfpathmoveto{\pgfqpoint{2.151081in}{0.502083in}}%
\pgfpathlineto{\pgfqpoint{2.602353in}{0.502083in}}%
\pgfpathlineto{\pgfqpoint{2.602353in}{2.518529in}}%
\pgfpathlineto{\pgfqpoint{2.151081in}{2.518529in}}%
\pgfpathlineto{\pgfqpoint{2.151081in}{0.502083in}}%
\pgfpathclose%
\pgfusepath{stroke,fill}%
\end{pgfscope}%
\begin{pgfscope}%
\pgfpathrectangle{\pgfqpoint{0.554706in}{0.502083in}}{\pgfqpoint{4.095294in}{2.347917in}}%
\pgfusepath{clip}%
\pgfsetbuttcap%
\pgfsetmiterjoin%
\definecolor{currentfill}{rgb}{0.172549,0.627451,0.172549}%
\pgfsetfillcolor{currentfill}%
\pgfsetlinewidth{0.401500pt}%
\definecolor{currentstroke}{rgb}{0.000000,0.000000,0.000000}%
\pgfsetstrokecolor{currentstroke}%
\pgfsetdash{}{0pt}%
\pgfpathmoveto{\pgfqpoint{3.561306in}{0.502083in}}%
\pgfpathlineto{\pgfqpoint{4.012578in}{0.502083in}}%
\pgfpathlineto{\pgfqpoint{4.012578in}{2.408039in}}%
\pgfpathlineto{\pgfqpoint{3.561306in}{2.408039in}}%
\pgfpathlineto{\pgfqpoint{3.561306in}{0.502083in}}%
\pgfpathclose%
\pgfusepath{stroke,fill}%
\end{pgfscope}%
\begin{pgfscope}%
\pgfpathrectangle{\pgfqpoint{0.554706in}{0.502083in}}{\pgfqpoint{4.095294in}{2.347917in}}%
\pgfusepath{clip}%
\pgfsetbuttcap%
\pgfsetmiterjoin%
\definecolor{currentfill}{rgb}{0.839216,0.152941,0.156863}%
\pgfsetfillcolor{currentfill}%
\pgfsetlinewidth{0.401500pt}%
\definecolor{currentstroke}{rgb}{0.000000,0.000000,0.000000}%
\pgfsetstrokecolor{currentstroke}%
\pgfsetdash{}{0pt}%
\pgfpathmoveto{\pgfqpoint{1.192128in}{0.502083in}}%
\pgfpathlineto{\pgfqpoint{1.643400in}{0.502083in}}%
\pgfpathlineto{\pgfqpoint{1.643400in}{1.303137in}}%
\pgfpathlineto{\pgfqpoint{1.192128in}{1.303137in}}%
\pgfpathlineto{\pgfqpoint{1.192128in}{0.502083in}}%
\pgfpathclose%
\pgfusepath{stroke,fill}%
\end{pgfscope}%
\begin{pgfscope}%
\pgfpathrectangle{\pgfqpoint{0.554706in}{0.502083in}}{\pgfqpoint{4.095294in}{2.347917in}}%
\pgfusepath{clip}%
\pgfsetbuttcap%
\pgfsetmiterjoin%
\definecolor{currentfill}{rgb}{0.839216,0.152941,0.156863}%
\pgfsetfillcolor{currentfill}%
\pgfsetlinewidth{0.401500pt}%
\definecolor{currentstroke}{rgb}{0.000000,0.000000,0.000000}%
\pgfsetstrokecolor{currentstroke}%
\pgfsetdash{}{0pt}%
\pgfpathmoveto{\pgfqpoint{2.602353in}{0.502083in}}%
\pgfpathlineto{\pgfqpoint{3.053625in}{0.502083in}}%
\pgfpathlineto{\pgfqpoint{3.053625in}{0.502083in}}%
\pgfpathlineto{\pgfqpoint{2.602353in}{0.502083in}}%
\pgfpathlineto{\pgfqpoint{2.602353in}{0.502083in}}%
\pgfpathclose%
\pgfusepath{stroke,fill}%
\end{pgfscope}%
\begin{pgfscope}%
\pgfpathrectangle{\pgfqpoint{0.554706in}{0.502083in}}{\pgfqpoint{4.095294in}{2.347917in}}%
\pgfusepath{clip}%
\pgfsetbuttcap%
\pgfsetmiterjoin%
\definecolor{currentfill}{rgb}{0.839216,0.152941,0.156863}%
\pgfsetfillcolor{currentfill}%
\pgfsetlinewidth{0.401500pt}%
\definecolor{currentstroke}{rgb}{0.000000,0.000000,0.000000}%
\pgfsetstrokecolor{currentstroke}%
\pgfsetdash{}{0pt}%
\pgfpathmoveto{\pgfqpoint{4.012578in}{0.502083in}}%
\pgfpathlineto{\pgfqpoint{4.463850in}{0.502083in}}%
\pgfpathlineto{\pgfqpoint{4.463850in}{0.612574in}}%
\pgfpathlineto{\pgfqpoint{4.012578in}{0.612574in}}%
\pgfpathlineto{\pgfqpoint{4.012578in}{0.502083in}}%
\pgfpathclose%
\pgfusepath{stroke,fill}%
\end{pgfscope}%
\begin{pgfscope}%
\pgfsetbuttcap%
\pgfsetroundjoin%
\definecolor{currentfill}{rgb}{0.000000,0.000000,0.000000}%
\pgfsetfillcolor{currentfill}%
\pgfsetlinewidth{0.803000pt}%
\definecolor{currentstroke}{rgb}{0.000000,0.000000,0.000000}%
\pgfsetstrokecolor{currentstroke}%
\pgfsetdash{}{0pt}%
\pgfsys@defobject{currentmarker}{\pgfqpoint{0.000000in}{-0.048611in}}{\pgfqpoint{0.000000in}{0.000000in}}{%
\pgfpathmoveto{\pgfqpoint{0.000000in}{0.000000in}}%
\pgfpathlineto{\pgfqpoint{0.000000in}{-0.048611in}}%
\pgfusepath{stroke,fill}%
}%
\begin{pgfscope}%
\pgfsys@transformshift{1.192128in}{0.502083in}%
\pgfsys@useobject{currentmarker}{}%
\end{pgfscope}%
\end{pgfscope}%
\begin{pgfscope}%
\definecolor{textcolor}{rgb}{0.000000,0.000000,0.000000}%
\pgfsetstrokecolor{textcolor}%
\pgfsetfillcolor{textcolor}%
\pgftext[x=1.005526in, y=0.318056in, left, base]{\color{textcolor}{\rmfamily\fontsize{9.000000}{10.800000}\selectfont\catcode`\^=\active\def^{\ifmmode\sp\else\^{}\fi}\catcode`\%=\active\def%{\%}Limiar}}%
\end{pgfscope}%
\begin{pgfscope}%
\definecolor{textcolor}{rgb}{0.000000,0.000000,0.000000}%
\pgfsetstrokecolor{textcolor}%
\pgfsetfillcolor{textcolor}%
\pgftext[x=1.060088in, y=0.181250in, left, base]{\color{textcolor}{\rmfamily\fontsize{9.000000}{10.800000}\selectfont\catcode`\^=\active\def^{\ifmmode\sp\else\^{}\fi}\catcode`\%=\active\def%{\%}($\geq$8)}}%
\end{pgfscope}%
\begin{pgfscope}%
\pgfsetbuttcap%
\pgfsetroundjoin%
\definecolor{currentfill}{rgb}{0.000000,0.000000,0.000000}%
\pgfsetfillcolor{currentfill}%
\pgfsetlinewidth{0.803000pt}%
\definecolor{currentstroke}{rgb}{0.000000,0.000000,0.000000}%
\pgfsetstrokecolor{currentstroke}%
\pgfsetdash{}{0pt}%
\pgfsys@defobject{currentmarker}{\pgfqpoint{0.000000in}{-0.048611in}}{\pgfqpoint{0.000000in}{0.000000in}}{%
\pgfpathmoveto{\pgfqpoint{0.000000in}{0.000000in}}%
\pgfpathlineto{\pgfqpoint{0.000000in}{-0.048611in}}%
\pgfusepath{stroke,fill}%
}%
\begin{pgfscope}%
\pgfsys@transformshift{2.602353in}{0.502083in}%
\pgfsys@useobject{currentmarker}{}%
\end{pgfscope}%
\end{pgfscope}%
\begin{pgfscope}%
\definecolor{textcolor}{rgb}{0.000000,0.000000,0.000000}%
\pgfsetstrokecolor{textcolor}%
\pgfsetfillcolor{textcolor}%
\pgftext[x=2.362376in, y=0.318056in, left, base]{\color{textcolor}{\rmfamily\fontsize{9.000000}{10.800000}\selectfont\catcode`\^=\active\def^{\ifmmode\sp\else\^{}\fi}\catcode`\%=\active\def%{\%}ML-only}}%
\end{pgfscope}%
\begin{pgfscope}%
\definecolor{textcolor}{rgb}{0.000000,0.000000,0.000000}%
\pgfsetstrokecolor{textcolor}%
\pgfsetfillcolor{textcolor}%
\pgftext[x=2.228561in, y=0.189583in, left, base]{\color{textcolor}{\rmfamily\fontsize{9.000000}{10.800000}\selectfont\catcode`\^=\active\def^{\ifmmode\sp\else\^{}\fi}\catcode`\%=\active\def%{\%}RF+SMOTE}}%
\end{pgfscope}%
\begin{pgfscope}%
\pgfsetbuttcap%
\pgfsetroundjoin%
\definecolor{currentfill}{rgb}{0.000000,0.000000,0.000000}%
\pgfsetfillcolor{currentfill}%
\pgfsetlinewidth{0.803000pt}%
\definecolor{currentstroke}{rgb}{0.000000,0.000000,0.000000}%
\pgfsetstrokecolor{currentstroke}%
\pgfsetdash{}{0pt}%
\pgfsys@defobject{currentmarker}{\pgfqpoint{0.000000in}{-0.048611in}}{\pgfqpoint{0.000000in}{0.000000in}}{%
\pgfpathmoveto{\pgfqpoint{0.000000in}{0.000000in}}%
\pgfpathlineto{\pgfqpoint{0.000000in}{-0.048611in}}%
\pgfusepath{stroke,fill}%
}%
\begin{pgfscope}%
\pgfsys@transformshift{4.012578in}{0.502083in}%
\pgfsys@useobject{currentmarker}{}%
\end{pgfscope}%
\end{pgfscope}%
\begin{pgfscope}%
\definecolor{textcolor}{rgb}{0.000000,0.000000,0.000000}%
\pgfsetstrokecolor{textcolor}%
\pgfsetfillcolor{textcolor}%
\pgftext[x=3.697515in, y=0.318056in, left, base]{\color{textcolor}{\rmfamily\fontsize{9.000000}{10.800000}\selectfont\catcode`\^=\active\def^{\ifmmode\sp\else\^{}\fi}\catcode`\%=\active\def%{\%}Hybrid-OR}}%
\end{pgfscope}%
\begin{pgfscope}%
\definecolor{textcolor}{rgb}{0.000000,0.000000,0.000000}%
\pgfsetstrokecolor{textcolor}%
\pgfsetfillcolor{textcolor}%
\pgftext[x=3.718691in, y=0.189583in, left, base]{\color{textcolor}{\rmfamily\fontsize{9.000000}{10.800000}\selectfont\catcode`\^=\active\def^{\ifmmode\sp\else\^{}\fi}\catcode`\%=\active\def%{\%}RF+Rules}}%
\end{pgfscope}%
\begin{pgfscope}%
\pgfsetbuttcap%
\pgfsetroundjoin%
\definecolor{currentfill}{rgb}{0.000000,0.000000,0.000000}%
\pgfsetfillcolor{currentfill}%
\pgfsetlinewidth{0.803000pt}%
\definecolor{currentstroke}{rgb}{0.000000,0.000000,0.000000}%
\pgfsetstrokecolor{currentstroke}%
\pgfsetdash{}{0pt}%
\pgfsys@defobject{currentmarker}{\pgfqpoint{-0.048611in}{0.000000in}}{\pgfqpoint{-0.000000in}{0.000000in}}{%
\pgfpathmoveto{\pgfqpoint{-0.000000in}{0.000000in}}%
\pgfpathlineto{\pgfqpoint{-0.048611in}{0.000000in}}%
\pgfusepath{stroke,fill}%
}%
\begin{pgfscope}%
\pgfsys@transformshift{0.554706in}{0.502083in}%
\pgfsys@useobject{currentmarker}{}%
\end{pgfscope}%
\end{pgfscope}%
\begin{pgfscope}%
\definecolor{textcolor}{rgb}{0.000000,0.000000,0.000000}%
\pgfsetstrokecolor{textcolor}%
\pgfsetfillcolor{textcolor}%
\pgftext[x=0.393248in, y=0.458681in, left, base]{\color{textcolor}{\rmfamily\fontsize{9.000000}{10.800000}\selectfont\catcode`\^=\active\def^{\ifmmode\sp\else\^{}\fi}\catcode`\%=\active\def%{\%}$\mathdefault{0}$}}%
\end{pgfscope}%
\begin{pgfscope}%
\pgfsetbuttcap%
\pgfsetroundjoin%
\definecolor{currentfill}{rgb}{0.000000,0.000000,0.000000}%
\pgfsetfillcolor{currentfill}%
\pgfsetlinewidth{0.803000pt}%
\definecolor{currentstroke}{rgb}{0.000000,0.000000,0.000000}%
\pgfsetstrokecolor{currentstroke}%
\pgfsetdash{}{0pt}%
\pgfsys@defobject{currentmarker}{\pgfqpoint{-0.048611in}{0.000000in}}{\pgfqpoint{-0.000000in}{0.000000in}}{%
\pgfpathmoveto{\pgfqpoint{-0.000000in}{0.000000in}}%
\pgfpathlineto{\pgfqpoint{-0.048611in}{0.000000in}}%
\pgfusepath{stroke,fill}%
}%
\begin{pgfscope}%
\pgfsys@transformshift{0.554706in}{0.778309in}%
\pgfsys@useobject{currentmarker}{}%
\end{pgfscope}%
\end{pgfscope}%
\begin{pgfscope}%
\definecolor{textcolor}{rgb}{0.000000,0.000000,0.000000}%
\pgfsetstrokecolor{textcolor}%
\pgfsetfillcolor{textcolor}%
\pgftext[x=0.329012in, y=0.734906in, left, base]{\color{textcolor}{\rmfamily\fontsize{9.000000}{10.800000}\selectfont\catcode`\^=\active\def^{\ifmmode\sp\else\^{}\fi}\catcode`\%=\active\def%{\%}$\mathdefault{10}$}}%
\end{pgfscope}%
\begin{pgfscope}%
\pgfsetbuttcap%
\pgfsetroundjoin%
\definecolor{currentfill}{rgb}{0.000000,0.000000,0.000000}%
\pgfsetfillcolor{currentfill}%
\pgfsetlinewidth{0.803000pt}%
\definecolor{currentstroke}{rgb}{0.000000,0.000000,0.000000}%
\pgfsetstrokecolor{currentstroke}%
\pgfsetdash{}{0pt}%
\pgfsys@defobject{currentmarker}{\pgfqpoint{-0.048611in}{0.000000in}}{\pgfqpoint{-0.000000in}{0.000000in}}{%
\pgfpathmoveto{\pgfqpoint{-0.000000in}{0.000000in}}%
\pgfpathlineto{\pgfqpoint{-0.048611in}{0.000000in}}%
\pgfusepath{stroke,fill}%
}%
\begin{pgfscope}%
\pgfsys@transformshift{0.554706in}{1.054534in}%
\pgfsys@useobject{currentmarker}{}%
\end{pgfscope}%
\end{pgfscope}%
\begin{pgfscope}%
\definecolor{textcolor}{rgb}{0.000000,0.000000,0.000000}%
\pgfsetstrokecolor{textcolor}%
\pgfsetfillcolor{textcolor}%
\pgftext[x=0.329012in, y=1.011132in, left, base]{\color{textcolor}{\rmfamily\fontsize{9.000000}{10.800000}\selectfont\catcode`\^=\active\def^{\ifmmode\sp\else\^{}\fi}\catcode`\%=\active\def%{\%}$\mathdefault{20}$}}%
\end{pgfscope}%
\begin{pgfscope}%
\pgfsetbuttcap%
\pgfsetroundjoin%
\definecolor{currentfill}{rgb}{0.000000,0.000000,0.000000}%
\pgfsetfillcolor{currentfill}%
\pgfsetlinewidth{0.803000pt}%
\definecolor{currentstroke}{rgb}{0.000000,0.000000,0.000000}%
\pgfsetstrokecolor{currentstroke}%
\pgfsetdash{}{0pt}%
\pgfsys@defobject{currentmarker}{\pgfqpoint{-0.048611in}{0.000000in}}{\pgfqpoint{-0.000000in}{0.000000in}}{%
\pgfpathmoveto{\pgfqpoint{-0.000000in}{0.000000in}}%
\pgfpathlineto{\pgfqpoint{-0.048611in}{0.000000in}}%
\pgfusepath{stroke,fill}%
}%
\begin{pgfscope}%
\pgfsys@transformshift{0.554706in}{1.330760in}%
\pgfsys@useobject{currentmarker}{}%
\end{pgfscope}%
\end{pgfscope}%
\begin{pgfscope}%
\definecolor{textcolor}{rgb}{0.000000,0.000000,0.000000}%
\pgfsetstrokecolor{textcolor}%
\pgfsetfillcolor{textcolor}%
\pgftext[x=0.329012in, y=1.287357in, left, base]{\color{textcolor}{\rmfamily\fontsize{9.000000}{10.800000}\selectfont\catcode`\^=\active\def^{\ifmmode\sp\else\^{}\fi}\catcode`\%=\active\def%{\%}$\mathdefault{30}$}}%
\end{pgfscope}%
\begin{pgfscope}%
\pgfsetbuttcap%
\pgfsetroundjoin%
\definecolor{currentfill}{rgb}{0.000000,0.000000,0.000000}%
\pgfsetfillcolor{currentfill}%
\pgfsetlinewidth{0.803000pt}%
\definecolor{currentstroke}{rgb}{0.000000,0.000000,0.000000}%
\pgfsetstrokecolor{currentstroke}%
\pgfsetdash{}{0pt}%
\pgfsys@defobject{currentmarker}{\pgfqpoint{-0.048611in}{0.000000in}}{\pgfqpoint{-0.000000in}{0.000000in}}{%
\pgfpathmoveto{\pgfqpoint{-0.000000in}{0.000000in}}%
\pgfpathlineto{\pgfqpoint{-0.048611in}{0.000000in}}%
\pgfusepath{stroke,fill}%
}%
\begin{pgfscope}%
\pgfsys@transformshift{0.554706in}{1.606985in}%
\pgfsys@useobject{currentmarker}{}%
\end{pgfscope}%
\end{pgfscope}%
\begin{pgfscope}%
\definecolor{textcolor}{rgb}{0.000000,0.000000,0.000000}%
\pgfsetstrokecolor{textcolor}%
\pgfsetfillcolor{textcolor}%
\pgftext[x=0.329012in, y=1.563583in, left, base]{\color{textcolor}{\rmfamily\fontsize{9.000000}{10.800000}\selectfont\catcode`\^=\active\def^{\ifmmode\sp\else\^{}\fi}\catcode`\%=\active\def%{\%}$\mathdefault{40}$}}%
\end{pgfscope}%
\begin{pgfscope}%
\pgfsetbuttcap%
\pgfsetroundjoin%
\definecolor{currentfill}{rgb}{0.000000,0.000000,0.000000}%
\pgfsetfillcolor{currentfill}%
\pgfsetlinewidth{0.803000pt}%
\definecolor{currentstroke}{rgb}{0.000000,0.000000,0.000000}%
\pgfsetstrokecolor{currentstroke}%
\pgfsetdash{}{0pt}%
\pgfsys@defobject{currentmarker}{\pgfqpoint{-0.048611in}{0.000000in}}{\pgfqpoint{-0.000000in}{0.000000in}}{%
\pgfpathmoveto{\pgfqpoint{-0.000000in}{0.000000in}}%
\pgfpathlineto{\pgfqpoint{-0.048611in}{0.000000in}}%
\pgfusepath{stroke,fill}%
}%
\begin{pgfscope}%
\pgfsys@transformshift{0.554706in}{1.883211in}%
\pgfsys@useobject{currentmarker}{}%
\end{pgfscope}%
\end{pgfscope}%
\begin{pgfscope}%
\definecolor{textcolor}{rgb}{0.000000,0.000000,0.000000}%
\pgfsetstrokecolor{textcolor}%
\pgfsetfillcolor{textcolor}%
\pgftext[x=0.329012in, y=1.839808in, left, base]{\color{textcolor}{\rmfamily\fontsize{9.000000}{10.800000}\selectfont\catcode`\^=\active\def^{\ifmmode\sp\else\^{}\fi}\catcode`\%=\active\def%{\%}$\mathdefault{50}$}}%
\end{pgfscope}%
\begin{pgfscope}%
\pgfsetbuttcap%
\pgfsetroundjoin%
\definecolor{currentfill}{rgb}{0.000000,0.000000,0.000000}%
\pgfsetfillcolor{currentfill}%
\pgfsetlinewidth{0.803000pt}%
\definecolor{currentstroke}{rgb}{0.000000,0.000000,0.000000}%
\pgfsetstrokecolor{currentstroke}%
\pgfsetdash{}{0pt}%
\pgfsys@defobject{currentmarker}{\pgfqpoint{-0.048611in}{0.000000in}}{\pgfqpoint{-0.000000in}{0.000000in}}{%
\pgfpathmoveto{\pgfqpoint{-0.000000in}{0.000000in}}%
\pgfpathlineto{\pgfqpoint{-0.048611in}{0.000000in}}%
\pgfusepath{stroke,fill}%
}%
\begin{pgfscope}%
\pgfsys@transformshift{0.554706in}{2.159436in}%
\pgfsys@useobject{currentmarker}{}%
\end{pgfscope}%
\end{pgfscope}%
\begin{pgfscope}%
\definecolor{textcolor}{rgb}{0.000000,0.000000,0.000000}%
\pgfsetstrokecolor{textcolor}%
\pgfsetfillcolor{textcolor}%
\pgftext[x=0.329012in, y=2.116033in, left, base]{\color{textcolor}{\rmfamily\fontsize{9.000000}{10.800000}\selectfont\catcode`\^=\active\def^{\ifmmode\sp\else\^{}\fi}\catcode`\%=\active\def%{\%}$\mathdefault{60}$}}%
\end{pgfscope}%
\begin{pgfscope}%
\pgfsetbuttcap%
\pgfsetroundjoin%
\definecolor{currentfill}{rgb}{0.000000,0.000000,0.000000}%
\pgfsetfillcolor{currentfill}%
\pgfsetlinewidth{0.803000pt}%
\definecolor{currentstroke}{rgb}{0.000000,0.000000,0.000000}%
\pgfsetstrokecolor{currentstroke}%
\pgfsetdash{}{0pt}%
\pgfsys@defobject{currentmarker}{\pgfqpoint{-0.048611in}{0.000000in}}{\pgfqpoint{-0.000000in}{0.000000in}}{%
\pgfpathmoveto{\pgfqpoint{-0.000000in}{0.000000in}}%
\pgfpathlineto{\pgfqpoint{-0.048611in}{0.000000in}}%
\pgfusepath{stroke,fill}%
}%
\begin{pgfscope}%
\pgfsys@transformshift{0.554706in}{2.435662in}%
\pgfsys@useobject{currentmarker}{}%
\end{pgfscope}%
\end{pgfscope}%
\begin{pgfscope}%
\definecolor{textcolor}{rgb}{0.000000,0.000000,0.000000}%
\pgfsetstrokecolor{textcolor}%
\pgfsetfillcolor{textcolor}%
\pgftext[x=0.329012in, y=2.392259in, left, base]{\color{textcolor}{\rmfamily\fontsize{9.000000}{10.800000}\selectfont\catcode`\^=\active\def^{\ifmmode\sp\else\^{}\fi}\catcode`\%=\active\def%{\%}$\mathdefault{70}$}}%
\end{pgfscope}%
\begin{pgfscope}%
\pgfsetbuttcap%
\pgfsetroundjoin%
\definecolor{currentfill}{rgb}{0.000000,0.000000,0.000000}%
\pgfsetfillcolor{currentfill}%
\pgfsetlinewidth{0.803000pt}%
\definecolor{currentstroke}{rgb}{0.000000,0.000000,0.000000}%
\pgfsetstrokecolor{currentstroke}%
\pgfsetdash{}{0pt}%
\pgfsys@defobject{currentmarker}{\pgfqpoint{-0.048611in}{0.000000in}}{\pgfqpoint{-0.000000in}{0.000000in}}{%
\pgfpathmoveto{\pgfqpoint{-0.000000in}{0.000000in}}%
\pgfpathlineto{\pgfqpoint{-0.048611in}{0.000000in}}%
\pgfusepath{stroke,fill}%
}%
\begin{pgfscope}%
\pgfsys@transformshift{0.554706in}{2.711887in}%
\pgfsys@useobject{currentmarker}{}%
\end{pgfscope}%
\end{pgfscope}%
\begin{pgfscope}%
\definecolor{textcolor}{rgb}{0.000000,0.000000,0.000000}%
\pgfsetstrokecolor{textcolor}%
\pgfsetfillcolor{textcolor}%
\pgftext[x=0.329012in, y=2.668484in, left, base]{\color{textcolor}{\rmfamily\fontsize{9.000000}{10.800000}\selectfont\catcode`\^=\active\def^{\ifmmode\sp\else\^{}\fi}\catcode`\%=\active\def%{\%}$\mathdefault{80}$}}%
\end{pgfscope}%
\begin{pgfscope}%
\definecolor{textcolor}{rgb}{0.000000,0.000000,0.000000}%
\pgfsetstrokecolor{textcolor}%
\pgfsetfillcolor{textcolor}%
\pgftext[x=0.273457in,y=1.676042in,,bottom,rotate=90.000000]{\color{textcolor}{\rmfamily\fontsize{10.000000}{12.000000}\selectfont\catcode`\^=\active\def^{\ifmmode\sp\else\^{}\fi}\catcode`\%=\active\def%{\%}N. de pares verdadeiros}}%
\end{pgfscope}%
\begin{pgfscope}%
\pgfsetrectcap%
\pgfsetmiterjoin%
\pgfsetlinewidth{0.803000pt}%
\definecolor{currentstroke}{rgb}{0.000000,0.000000,0.000000}%
\pgfsetstrokecolor{currentstroke}%
\pgfsetdash{}{0pt}%
\pgfpathmoveto{\pgfqpoint{0.554706in}{0.502083in}}%
\pgfpathlineto{\pgfqpoint{0.554706in}{2.850000in}}%
\pgfusepath{stroke}%
\end{pgfscope}%
\begin{pgfscope}%
\pgfsetrectcap%
\pgfsetmiterjoin%
\pgfsetlinewidth{0.803000pt}%
\definecolor{currentstroke}{rgb}{0.000000,0.000000,0.000000}%
\pgfsetstrokecolor{currentstroke}%
\pgfsetdash{}{0pt}%
\pgfpathmoveto{\pgfqpoint{0.554706in}{0.502083in}}%
\pgfpathlineto{\pgfqpoint{4.650000in}{0.502083in}}%
\pgfusepath{stroke}%
\end{pgfscope}%
\begin{pgfscope}%
\definecolor{textcolor}{rgb}{0.000000,0.000000,0.000000}%
\pgfsetstrokecolor{textcolor}%
\pgfsetfillcolor{textcolor}%
\pgftext[x=0.966492in,y=1.745098in,,base]{\color{textcolor}{\rmfamily\fontsize{9.000000}{10.800000}\selectfont\catcode`\^=\active\def^{\ifmmode\sp\else\^{}\fi}\catcode`\%=\active\def%{\%}44}}%
\end{pgfscope}%
\begin{pgfscope}%
\definecolor{textcolor}{rgb}{0.000000,0.000000,0.000000}%
\pgfsetstrokecolor{textcolor}%
\pgfsetfillcolor{textcolor}%
\pgftext[x=2.376717in,y=2.546152in,,base]{\color{textcolor}{\rmfamily\fontsize{9.000000}{10.800000}\selectfont\catcode`\^=\active\def^{\ifmmode\sp\else\^{}\fi}\catcode`\%=\active\def%{\%}73}}%
\end{pgfscope}%
\begin{pgfscope}%
\definecolor{textcolor}{rgb}{0.000000,0.000000,0.000000}%
\pgfsetstrokecolor{textcolor}%
\pgfsetfillcolor{textcolor}%
\pgftext[x=3.786942in,y=2.435662in,,base]{\color{textcolor}{\rmfamily\fontsize{9.000000}{10.800000}\selectfont\catcode`\^=\active\def^{\ifmmode\sp\else\^{}\fi}\catcode`\%=\active\def%{\%}69}}%
\end{pgfscope}%
\begin{pgfscope}%
\definecolor{textcolor}{rgb}{0.000000,0.000000,0.000000}%
\pgfsetstrokecolor{textcolor}%
\pgfsetfillcolor{textcolor}%
\pgftext[x=1.417764in,y=1.330760in,,base]{\color{textcolor}{\rmfamily\fontsize{9.000000}{10.800000}\selectfont\catcode`\^=\active\def^{\ifmmode\sp\else\^{}\fi}\catcode`\%=\active\def%{\%}29}}%
\end{pgfscope}%
\begin{pgfscope}%
\definecolor{textcolor}{rgb}{0.000000,0.000000,0.000000}%
\pgfsetstrokecolor{textcolor}%
\pgfsetfillcolor{textcolor}%
\pgftext[x=2.827989in,y=0.529706in,,base]{\color{textcolor}{\rmfamily\fontsize{9.000000}{10.800000}\selectfont\catcode`\^=\active\def^{\ifmmode\sp\else\^{}\fi}\catcode`\%=\active\def%{\%}0}}%
\end{pgfscope}%
\begin{pgfscope}%
\definecolor{textcolor}{rgb}{0.000000,0.000000,0.000000}%
\pgfsetstrokecolor{textcolor}%
\pgfsetfillcolor{textcolor}%
\pgftext[x=4.238214in,y=0.640196in,,base]{\color{textcolor}{\rmfamily\fontsize{9.000000}{10.800000}\selectfont\catcode`\^=\active\def^{\ifmmode\sp\else\^{}\fi}\catcode`\%=\active\def%{\%}4}}%
\end{pgfscope}%
\begin{pgfscope}%
\pgfsetbuttcap%
\pgfsetmiterjoin%
\definecolor{currentfill}{rgb}{1.000000,1.000000,1.000000}%
\pgfsetfillcolor{currentfill}%
\pgfsetfillopacity{0.900000}%
\pgfsetlinewidth{1.003750pt}%
\definecolor{currentstroke}{rgb}{0.800000,0.800000,0.800000}%
\pgfsetstrokecolor{currentstroke}%
\pgfsetstrokeopacity{0.900000}%
\pgfsetdash{}{0pt}%
\pgfpathmoveto{\pgfqpoint{0.642206in}{2.401389in}}%
\pgfpathlineto{\pgfqpoint{1.662167in}{2.401389in}}%
\pgfpathquadraticcurveto{\pgfqpoint{1.687167in}{2.401389in}}{\pgfqpoint{1.687167in}{2.426389in}}%
\pgfpathlineto{\pgfqpoint{1.687167in}{2.762500in}}%
\pgfpathquadraticcurveto{\pgfqpoint{1.687167in}{2.787500in}}{\pgfqpoint{1.662167in}{2.787500in}}%
\pgfpathlineto{\pgfqpoint{0.642206in}{2.787500in}}%
\pgfpathquadraticcurveto{\pgfqpoint{0.617206in}{2.787500in}}{\pgfqpoint{0.617206in}{2.762500in}}%
\pgfpathlineto{\pgfqpoint{0.617206in}{2.426389in}}%
\pgfpathquadraticcurveto{\pgfqpoint{0.617206in}{2.401389in}}{\pgfqpoint{0.642206in}{2.401389in}}%
\pgfpathlineto{\pgfqpoint{0.642206in}{2.401389in}}%
\pgfpathclose%
\pgfusepath{stroke,fill}%
\end{pgfscope}%
\begin{pgfscope}%
\pgfsetbuttcap%
\pgfsetmiterjoin%
\definecolor{currentfill}{rgb}{0.172549,0.627451,0.172549}%
\pgfsetfillcolor{currentfill}%
\pgfsetlinewidth{0.401500pt}%
\definecolor{currentstroke}{rgb}{0.000000,0.000000,0.000000}%
\pgfsetstrokecolor{currentstroke}%
\pgfsetdash{}{0pt}%
\pgfpathmoveto{\pgfqpoint{0.667206in}{2.650000in}}%
\pgfpathlineto{\pgfqpoint{0.917206in}{2.650000in}}%
\pgfpathlineto{\pgfqpoint{0.917206in}{2.737500in}}%
\pgfpathlineto{\pgfqpoint{0.667206in}{2.737500in}}%
\pgfpathlineto{\pgfqpoint{0.667206in}{2.650000in}}%
\pgfpathclose%
\pgfusepath{stroke,fill}%
\end{pgfscope}%
\begin{pgfscope}%
\definecolor{textcolor}{rgb}{0.000000,0.000000,0.000000}%
\pgfsetstrokecolor{textcolor}%
\pgfsetfillcolor{textcolor}%
\pgftext[x=1.017206in,y=2.650000in,left,base]{\color{textcolor}{\rmfamily\fontsize{9.000000}{10.800000}\selectfont\catcode`\^=\active\def^{\ifmmode\sp\else\^{}\fi}\catcode`\%=\active\def%{\%}Detectados}}%
\end{pgfscope}%
\begin{pgfscope}%
\pgfsetbuttcap%
\pgfsetmiterjoin%
\definecolor{currentfill}{rgb}{0.839216,0.152941,0.156863}%
\pgfsetfillcolor{currentfill}%
\pgfsetlinewidth{0.401500pt}%
\definecolor{currentstroke}{rgb}{0.000000,0.000000,0.000000}%
\pgfsetstrokecolor{currentstroke}%
\pgfsetdash{}{0pt}%
\pgfpathmoveto{\pgfqpoint{0.667206in}{2.475694in}}%
\pgfpathlineto{\pgfqpoint{0.917206in}{2.475694in}}%
\pgfpathlineto{\pgfqpoint{0.917206in}{2.563194in}}%
\pgfpathlineto{\pgfqpoint{0.667206in}{2.563194in}}%
\pgfpathlineto{\pgfqpoint{0.667206in}{2.475694in}}%
\pgfpathclose%
\pgfusepath{stroke,fill}%
\end{pgfscope}%
\begin{pgfscope}%
\definecolor{textcolor}{rgb}{0.000000,0.000000,0.000000}%
\pgfsetstrokecolor{textcolor}%
\pgfsetfillcolor{textcolor}%
\pgftext[x=1.017206in,y=2.475694in,left,base]{\color{textcolor}{\rmfamily\fontsize{9.000000}{10.800000}\selectfont\catcode`\^=\active\def^{\ifmmode\sp\else\^{}\fi}\catcode`\%=\active\def%{\%}Perdidos}}%
\end{pgfscope}%
\end{pgfpicture}%
\makeatother%
\endgroup%

\caption{Pares verdadeiros detectados e perdidos por método de classificação.}
\label{fig:epi-deteccao}
\end{figure}

\begin{figure}[!ht]
\centering
%% Creator: Matplotlib, PGF backend
%%
%% To include the figure in your LaTeX document, write
%%   \input{<filename>.pgf}
%%
%% Make sure the required packages are loaded in your preamble
%%   \usepackage{pgf}
%%
%% Also ensure that all the required font packages are loaded; for instance,
%% the lmodern package is sometimes necessary when using math font.
%%   \usepackage{lmodern}
%%
%% Figures using additional raster images can only be included by \input if
%% they are in the same directory as the main LaTeX file. For loading figures
%% from other directories you can use the `import` package
%%   \usepackage{import}
%%
%% and then include the figures with
%%   \import{<path to file>}{<filename>.pgf}
%%
%% Matplotlib used the following preamble
%%   \def\mathdefault#1{#1}
%%   \everymath=\expandafter{\the\everymath\displaystyle}
%%   \IfFileExists{scrextend.sty}{
%%     \usepackage[fontsize=10.000000pt]{scrextend}
%%   }{
%%     \renewcommand{\normalsize}{\fontsize{10.000000}{12.000000}\selectfont}
%%     \normalsize
%%   }
%%   
%%   \makeatletter\@ifpackageloaded{underscore}{}{\usepackage[strings]{underscore}}\makeatother
%%
\begingroup%
\makeatletter%
\begin{pgfpicture}%
\pgfpathrectangle{\pgfpointorigin}{\pgfqpoint{4.800000in}{3.000000in}}%
\pgfusepath{use as bounding box, clip}%
\begin{pgfscope}%
\pgfsetbuttcap%
\pgfsetmiterjoin%
\definecolor{currentfill}{rgb}{1.000000,1.000000,1.000000}%
\pgfsetfillcolor{currentfill}%
\pgfsetlinewidth{0.000000pt}%
\definecolor{currentstroke}{rgb}{1.000000,1.000000,1.000000}%
\pgfsetstrokecolor{currentstroke}%
\pgfsetdash{}{0pt}%
\pgfpathmoveto{\pgfqpoint{0.000000in}{0.000000in}}%
\pgfpathlineto{\pgfqpoint{4.800000in}{0.000000in}}%
\pgfpathlineto{\pgfqpoint{4.800000in}{3.000000in}}%
\pgfpathlineto{\pgfqpoint{0.000000in}{3.000000in}}%
\pgfpathlineto{\pgfqpoint{0.000000in}{0.000000in}}%
\pgfpathclose%
\pgfusepath{fill}%
\end{pgfscope}%
\begin{pgfscope}%
\pgfsetbuttcap%
\pgfsetmiterjoin%
\definecolor{currentfill}{rgb}{1.000000,1.000000,1.000000}%
\pgfsetfillcolor{currentfill}%
\pgfsetlinewidth{0.000000pt}%
\definecolor{currentstroke}{rgb}{0.000000,0.000000,0.000000}%
\pgfsetstrokecolor{currentstroke}%
\pgfsetstrokeopacity{0.000000}%
\pgfsetdash{}{0pt}%
\pgfpathmoveto{\pgfqpoint{0.618942in}{0.502083in}}%
\pgfpathlineto{\pgfqpoint{4.650000in}{0.502083in}}%
\pgfpathlineto{\pgfqpoint{4.650000in}{2.850000in}}%
\pgfpathlineto{\pgfqpoint{0.618942in}{2.850000in}}%
\pgfpathlineto{\pgfqpoint{0.618942in}{0.502083in}}%
\pgfpathclose%
\pgfusepath{fill}%
\end{pgfscope}%
\begin{pgfscope}%
\pgfpathrectangle{\pgfqpoint{0.618942in}{0.502083in}}{\pgfqpoint{4.031058in}{2.347917in}}%
\pgfusepath{clip}%
\pgfsetbuttcap%
\pgfsetmiterjoin%
\definecolor{currentfill}{rgb}{1.000000,0.498039,0.054902}%
\pgfsetfillcolor{currentfill}%
\pgfsetlinewidth{0.401500pt}%
\definecolor{currentstroke}{rgb}{0.000000,0.000000,0.000000}%
\pgfsetstrokecolor{currentstroke}%
\pgfsetdash{}{0pt}%
\pgfpathmoveto{\pgfqpoint{0.802171in}{0.502083in}}%
\pgfpathlineto{\pgfqpoint{1.475261in}{0.502083in}}%
\pgfpathlineto{\pgfqpoint{1.475261in}{2.510856in}}%
\pgfpathlineto{\pgfqpoint{0.802171in}{2.510856in}}%
\pgfpathlineto{\pgfqpoint{0.802171in}{0.502083in}}%
\pgfpathclose%
\pgfusepath{stroke,fill}%
\end{pgfscope}%
\begin{pgfscope}%
\pgfpathrectangle{\pgfqpoint{0.618942in}{0.502083in}}{\pgfqpoint{4.031058in}{2.347917in}}%
\pgfusepath{clip}%
\pgfsetbuttcap%
\pgfsetmiterjoin%
\definecolor{currentfill}{rgb}{0.172549,0.627451,0.172549}%
\pgfsetfillcolor{currentfill}%
\pgfsetlinewidth{0.401500pt}%
\definecolor{currentstroke}{rgb}{0.000000,0.000000,0.000000}%
\pgfsetstrokecolor{currentstroke}%
\pgfsetdash{}{0pt}%
\pgfpathmoveto{\pgfqpoint{2.297926in}{0.502083in}}%
\pgfpathlineto{\pgfqpoint{2.971016in}{0.502083in}}%
\pgfpathlineto{\pgfqpoint{2.971016in}{1.023843in}}%
\pgfpathlineto{\pgfqpoint{2.297926in}{1.023843in}}%
\pgfpathlineto{\pgfqpoint{2.297926in}{0.502083in}}%
\pgfpathclose%
\pgfusepath{stroke,fill}%
\end{pgfscope}%
\begin{pgfscope}%
\pgfpathrectangle{\pgfqpoint{0.618942in}{0.502083in}}{\pgfqpoint{4.031058in}{2.347917in}}%
\pgfusepath{clip}%
\pgfsetbuttcap%
\pgfsetmiterjoin%
\definecolor{currentfill}{rgb}{0.090196,0.745098,0.811765}%
\pgfsetfillcolor{currentfill}%
\pgfsetlinewidth{0.401500pt}%
\definecolor{currentstroke}{rgb}{0.000000,0.000000,0.000000}%
\pgfsetstrokecolor{currentstroke}%
\pgfsetdash{}{0pt}%
\pgfpathmoveto{\pgfqpoint{3.793681in}{0.502083in}}%
\pgfpathlineto{\pgfqpoint{4.466770in}{0.502083in}}%
\pgfpathlineto{\pgfqpoint{4.466770in}{1.121672in}}%
\pgfpathlineto{\pgfqpoint{3.793681in}{1.121672in}}%
\pgfpathlineto{\pgfqpoint{3.793681in}{0.502083in}}%
\pgfpathclose%
\pgfusepath{stroke,fill}%
\end{pgfscope}%
\begin{pgfscope}%
\pgfsetbuttcap%
\pgfsetroundjoin%
\definecolor{currentfill}{rgb}{0.000000,0.000000,0.000000}%
\pgfsetfillcolor{currentfill}%
\pgfsetlinewidth{0.803000pt}%
\definecolor{currentstroke}{rgb}{0.000000,0.000000,0.000000}%
\pgfsetstrokecolor{currentstroke}%
\pgfsetdash{}{0pt}%
\pgfsys@defobject{currentmarker}{\pgfqpoint{0.000000in}{-0.048611in}}{\pgfqpoint{0.000000in}{0.000000in}}{%
\pgfpathmoveto{\pgfqpoint{0.000000in}{0.000000in}}%
\pgfpathlineto{\pgfqpoint{0.000000in}{-0.048611in}}%
\pgfusepath{stroke,fill}%
}%
\begin{pgfscope}%
\pgfsys@transformshift{1.138716in}{0.502083in}%
\pgfsys@useobject{currentmarker}{}%
\end{pgfscope}%
\end{pgfscope}%
\begin{pgfscope}%
\definecolor{textcolor}{rgb}{0.000000,0.000000,0.000000}%
\pgfsetstrokecolor{textcolor}%
\pgfsetfillcolor{textcolor}%
\pgftext[x=0.952115in, y=0.318056in, left, base]{\color{textcolor}{\rmfamily\fontsize{9.000000}{10.800000}\selectfont\catcode`\^=\active\def^{\ifmmode\sp\else\^{}\fi}\catcode`\%=\active\def%{\%}Limiar}}%
\end{pgfscope}%
\begin{pgfscope}%
\definecolor{textcolor}{rgb}{0.000000,0.000000,0.000000}%
\pgfsetstrokecolor{textcolor}%
\pgfsetfillcolor{textcolor}%
\pgftext[x=1.006676in, y=0.181250in, left, base]{\color{textcolor}{\rmfamily\fontsize{9.000000}{10.800000}\selectfont\catcode`\^=\active\def^{\ifmmode\sp\else\^{}\fi}\catcode`\%=\active\def%{\%}($\geq$8)}}%
\end{pgfscope}%
\begin{pgfscope}%
\pgfsetbuttcap%
\pgfsetroundjoin%
\definecolor{currentfill}{rgb}{0.000000,0.000000,0.000000}%
\pgfsetfillcolor{currentfill}%
\pgfsetlinewidth{0.803000pt}%
\definecolor{currentstroke}{rgb}{0.000000,0.000000,0.000000}%
\pgfsetstrokecolor{currentstroke}%
\pgfsetdash{}{0pt}%
\pgfsys@defobject{currentmarker}{\pgfqpoint{0.000000in}{-0.048611in}}{\pgfqpoint{0.000000in}{0.000000in}}{%
\pgfpathmoveto{\pgfqpoint{0.000000in}{0.000000in}}%
\pgfpathlineto{\pgfqpoint{0.000000in}{-0.048611in}}%
\pgfusepath{stroke,fill}%
}%
\begin{pgfscope}%
\pgfsys@transformshift{2.634471in}{0.502083in}%
\pgfsys@useobject{currentmarker}{}%
\end{pgfscope}%
\end{pgfscope}%
\begin{pgfscope}%
\definecolor{textcolor}{rgb}{0.000000,0.000000,0.000000}%
\pgfsetstrokecolor{textcolor}%
\pgfsetfillcolor{textcolor}%
\pgftext[x=2.394494in, y=0.318056in, left, base]{\color{textcolor}{\rmfamily\fontsize{9.000000}{10.800000}\selectfont\catcode`\^=\active\def^{\ifmmode\sp\else\^{}\fi}\catcode`\%=\active\def%{\%}ML-only}}%
\end{pgfscope}%
\begin{pgfscope}%
\definecolor{textcolor}{rgb}{0.000000,0.000000,0.000000}%
\pgfsetstrokecolor{textcolor}%
\pgfsetfillcolor{textcolor}%
\pgftext[x=2.260679in, y=0.189583in, left, base]{\color{textcolor}{\rmfamily\fontsize{9.000000}{10.800000}\selectfont\catcode`\^=\active\def^{\ifmmode\sp\else\^{}\fi}\catcode`\%=\active\def%{\%}RF+SMOTE}}%
\end{pgfscope}%
\begin{pgfscope}%
\pgfsetbuttcap%
\pgfsetroundjoin%
\definecolor{currentfill}{rgb}{0.000000,0.000000,0.000000}%
\pgfsetfillcolor{currentfill}%
\pgfsetlinewidth{0.803000pt}%
\definecolor{currentstroke}{rgb}{0.000000,0.000000,0.000000}%
\pgfsetstrokecolor{currentstroke}%
\pgfsetdash{}{0pt}%
\pgfsys@defobject{currentmarker}{\pgfqpoint{0.000000in}{-0.048611in}}{\pgfqpoint{0.000000in}{0.000000in}}{%
\pgfpathmoveto{\pgfqpoint{0.000000in}{0.000000in}}%
\pgfpathlineto{\pgfqpoint{0.000000in}{-0.048611in}}%
\pgfusepath{stroke,fill}%
}%
\begin{pgfscope}%
\pgfsys@transformshift{4.130225in}{0.502083in}%
\pgfsys@useobject{currentmarker}{}%
\end{pgfscope}%
\end{pgfscope}%
\begin{pgfscope}%
\definecolor{textcolor}{rgb}{0.000000,0.000000,0.000000}%
\pgfsetstrokecolor{textcolor}%
\pgfsetfillcolor{textcolor}%
\pgftext[x=3.815162in, y=0.318056in, left, base]{\color{textcolor}{\rmfamily\fontsize{9.000000}{10.800000}\selectfont\catcode`\^=\active\def^{\ifmmode\sp\else\^{}\fi}\catcode`\%=\active\def%{\%}Hybrid-OR}}%
\end{pgfscope}%
\begin{pgfscope}%
\definecolor{textcolor}{rgb}{0.000000,0.000000,0.000000}%
\pgfsetstrokecolor{textcolor}%
\pgfsetfillcolor{textcolor}%
\pgftext[x=3.836338in, y=0.189583in, left, base]{\color{textcolor}{\rmfamily\fontsize{9.000000}{10.800000}\selectfont\catcode`\^=\active\def^{\ifmmode\sp\else\^{}\fi}\catcode`\%=\active\def%{\%}RF+Rules}}%
\end{pgfscope}%
\begin{pgfscope}%
\pgfsetbuttcap%
\pgfsetroundjoin%
\definecolor{currentfill}{rgb}{0.000000,0.000000,0.000000}%
\pgfsetfillcolor{currentfill}%
\pgfsetlinewidth{0.803000pt}%
\definecolor{currentstroke}{rgb}{0.000000,0.000000,0.000000}%
\pgfsetstrokecolor{currentstroke}%
\pgfsetdash{}{0pt}%
\pgfsys@defobject{currentmarker}{\pgfqpoint{-0.048611in}{0.000000in}}{\pgfqpoint{-0.000000in}{0.000000in}}{%
\pgfpathmoveto{\pgfqpoint{-0.000000in}{0.000000in}}%
\pgfpathlineto{\pgfqpoint{-0.048611in}{0.000000in}}%
\pgfusepath{stroke,fill}%
}%
\begin{pgfscope}%
\pgfsys@transformshift{0.618942in}{0.502083in}%
\pgfsys@useobject{currentmarker}{}%
\end{pgfscope}%
\end{pgfscope}%
\begin{pgfscope}%
\definecolor{textcolor}{rgb}{0.000000,0.000000,0.000000}%
\pgfsetstrokecolor{textcolor}%
\pgfsetfillcolor{textcolor}%
\pgftext[x=0.457484in, y=0.458681in, left, base]{\color{textcolor}{\rmfamily\fontsize{9.000000}{10.800000}\selectfont\catcode`\^=\active\def^{\ifmmode\sp\else\^{}\fi}\catcode`\%=\active\def%{\%}$\mathdefault{0}$}}%
\end{pgfscope}%
\begin{pgfscope}%
\pgfsetbuttcap%
\pgfsetroundjoin%
\definecolor{currentfill}{rgb}{0.000000,0.000000,0.000000}%
\pgfsetfillcolor{currentfill}%
\pgfsetlinewidth{0.803000pt}%
\definecolor{currentstroke}{rgb}{0.000000,0.000000,0.000000}%
\pgfsetstrokecolor{currentstroke}%
\pgfsetdash{}{0pt}%
\pgfsys@defobject{currentmarker}{\pgfqpoint{-0.048611in}{0.000000in}}{\pgfqpoint{-0.000000in}{0.000000in}}{%
\pgfpathmoveto{\pgfqpoint{-0.000000in}{0.000000in}}%
\pgfpathlineto{\pgfqpoint{-0.048611in}{0.000000in}}%
\pgfusepath{stroke,fill}%
}%
\begin{pgfscope}%
\pgfsys@transformshift{0.618942in}{0.828183in}%
\pgfsys@useobject{currentmarker}{}%
\end{pgfscope}%
\end{pgfscope}%
\begin{pgfscope}%
\definecolor{textcolor}{rgb}{0.000000,0.000000,0.000000}%
\pgfsetstrokecolor{textcolor}%
\pgfsetfillcolor{textcolor}%
\pgftext[x=0.393248in, y=0.784780in, left, base]{\color{textcolor}{\rmfamily\fontsize{9.000000}{10.800000}\selectfont\catcode`\^=\active\def^{\ifmmode\sp\else\^{}\fi}\catcode`\%=\active\def%{\%}$\mathdefault{50}$}}%
\end{pgfscope}%
\begin{pgfscope}%
\pgfsetbuttcap%
\pgfsetroundjoin%
\definecolor{currentfill}{rgb}{0.000000,0.000000,0.000000}%
\pgfsetfillcolor{currentfill}%
\pgfsetlinewidth{0.803000pt}%
\definecolor{currentstroke}{rgb}{0.000000,0.000000,0.000000}%
\pgfsetstrokecolor{currentstroke}%
\pgfsetdash{}{0pt}%
\pgfsys@defobject{currentmarker}{\pgfqpoint{-0.048611in}{0.000000in}}{\pgfqpoint{-0.000000in}{0.000000in}}{%
\pgfpathmoveto{\pgfqpoint{-0.000000in}{0.000000in}}%
\pgfpathlineto{\pgfqpoint{-0.048611in}{0.000000in}}%
\pgfusepath{stroke,fill}%
}%
\begin{pgfscope}%
\pgfsys@transformshift{0.618942in}{1.154282in}%
\pgfsys@useobject{currentmarker}{}%
\end{pgfscope}%
\end{pgfscope}%
\begin{pgfscope}%
\definecolor{textcolor}{rgb}{0.000000,0.000000,0.000000}%
\pgfsetstrokecolor{textcolor}%
\pgfsetfillcolor{textcolor}%
\pgftext[x=0.329012in, y=1.110880in, left, base]{\color{textcolor}{\rmfamily\fontsize{9.000000}{10.800000}\selectfont\catcode`\^=\active\def^{\ifmmode\sp\else\^{}\fi}\catcode`\%=\active\def%{\%}$\mathdefault{100}$}}%
\end{pgfscope}%
\begin{pgfscope}%
\pgfsetbuttcap%
\pgfsetroundjoin%
\definecolor{currentfill}{rgb}{0.000000,0.000000,0.000000}%
\pgfsetfillcolor{currentfill}%
\pgfsetlinewidth{0.803000pt}%
\definecolor{currentstroke}{rgb}{0.000000,0.000000,0.000000}%
\pgfsetstrokecolor{currentstroke}%
\pgfsetdash{}{0pt}%
\pgfsys@defobject{currentmarker}{\pgfqpoint{-0.048611in}{0.000000in}}{\pgfqpoint{-0.000000in}{0.000000in}}{%
\pgfpathmoveto{\pgfqpoint{-0.000000in}{0.000000in}}%
\pgfpathlineto{\pgfqpoint{-0.048611in}{0.000000in}}%
\pgfusepath{stroke,fill}%
}%
\begin{pgfscope}%
\pgfsys@transformshift{0.618942in}{1.480382in}%
\pgfsys@useobject{currentmarker}{}%
\end{pgfscope}%
\end{pgfscope}%
\begin{pgfscope}%
\definecolor{textcolor}{rgb}{0.000000,0.000000,0.000000}%
\pgfsetstrokecolor{textcolor}%
\pgfsetfillcolor{textcolor}%
\pgftext[x=0.329012in, y=1.436979in, left, base]{\color{textcolor}{\rmfamily\fontsize{9.000000}{10.800000}\selectfont\catcode`\^=\active\def^{\ifmmode\sp\else\^{}\fi}\catcode`\%=\active\def%{\%}$\mathdefault{150}$}}%
\end{pgfscope}%
\begin{pgfscope}%
\pgfsetbuttcap%
\pgfsetroundjoin%
\definecolor{currentfill}{rgb}{0.000000,0.000000,0.000000}%
\pgfsetfillcolor{currentfill}%
\pgfsetlinewidth{0.803000pt}%
\definecolor{currentstroke}{rgb}{0.000000,0.000000,0.000000}%
\pgfsetstrokecolor{currentstroke}%
\pgfsetdash{}{0pt}%
\pgfsys@defobject{currentmarker}{\pgfqpoint{-0.048611in}{0.000000in}}{\pgfqpoint{-0.000000in}{0.000000in}}{%
\pgfpathmoveto{\pgfqpoint{-0.000000in}{0.000000in}}%
\pgfpathlineto{\pgfqpoint{-0.048611in}{0.000000in}}%
\pgfusepath{stroke,fill}%
}%
\begin{pgfscope}%
\pgfsys@transformshift{0.618942in}{1.806481in}%
\pgfsys@useobject{currentmarker}{}%
\end{pgfscope}%
\end{pgfscope}%
\begin{pgfscope}%
\definecolor{textcolor}{rgb}{0.000000,0.000000,0.000000}%
\pgfsetstrokecolor{textcolor}%
\pgfsetfillcolor{textcolor}%
\pgftext[x=0.329012in, y=1.763079in, left, base]{\color{textcolor}{\rmfamily\fontsize{9.000000}{10.800000}\selectfont\catcode`\^=\active\def^{\ifmmode\sp\else\^{}\fi}\catcode`\%=\active\def%{\%}$\mathdefault{200}$}}%
\end{pgfscope}%
\begin{pgfscope}%
\pgfsetbuttcap%
\pgfsetroundjoin%
\definecolor{currentfill}{rgb}{0.000000,0.000000,0.000000}%
\pgfsetfillcolor{currentfill}%
\pgfsetlinewidth{0.803000pt}%
\definecolor{currentstroke}{rgb}{0.000000,0.000000,0.000000}%
\pgfsetstrokecolor{currentstroke}%
\pgfsetdash{}{0pt}%
\pgfsys@defobject{currentmarker}{\pgfqpoint{-0.048611in}{0.000000in}}{\pgfqpoint{-0.000000in}{0.000000in}}{%
\pgfpathmoveto{\pgfqpoint{-0.000000in}{0.000000in}}%
\pgfpathlineto{\pgfqpoint{-0.048611in}{0.000000in}}%
\pgfusepath{stroke,fill}%
}%
\begin{pgfscope}%
\pgfsys@transformshift{0.618942in}{2.132581in}%
\pgfsys@useobject{currentmarker}{}%
\end{pgfscope}%
\end{pgfscope}%
\begin{pgfscope}%
\definecolor{textcolor}{rgb}{0.000000,0.000000,0.000000}%
\pgfsetstrokecolor{textcolor}%
\pgfsetfillcolor{textcolor}%
\pgftext[x=0.329012in, y=2.089178in, left, base]{\color{textcolor}{\rmfamily\fontsize{9.000000}{10.800000}\selectfont\catcode`\^=\active\def^{\ifmmode\sp\else\^{}\fi}\catcode`\%=\active\def%{\%}$\mathdefault{250}$}}%
\end{pgfscope}%
\begin{pgfscope}%
\pgfsetbuttcap%
\pgfsetroundjoin%
\definecolor{currentfill}{rgb}{0.000000,0.000000,0.000000}%
\pgfsetfillcolor{currentfill}%
\pgfsetlinewidth{0.803000pt}%
\definecolor{currentstroke}{rgb}{0.000000,0.000000,0.000000}%
\pgfsetstrokecolor{currentstroke}%
\pgfsetdash{}{0pt}%
\pgfsys@defobject{currentmarker}{\pgfqpoint{-0.048611in}{0.000000in}}{\pgfqpoint{-0.000000in}{0.000000in}}{%
\pgfpathmoveto{\pgfqpoint{-0.000000in}{0.000000in}}%
\pgfpathlineto{\pgfqpoint{-0.048611in}{0.000000in}}%
\pgfusepath{stroke,fill}%
}%
\begin{pgfscope}%
\pgfsys@transformshift{0.618942in}{2.458681in}%
\pgfsys@useobject{currentmarker}{}%
\end{pgfscope}%
\end{pgfscope}%
\begin{pgfscope}%
\definecolor{textcolor}{rgb}{0.000000,0.000000,0.000000}%
\pgfsetstrokecolor{textcolor}%
\pgfsetfillcolor{textcolor}%
\pgftext[x=0.329012in, y=2.415278in, left, base]{\color{textcolor}{\rmfamily\fontsize{9.000000}{10.800000}\selectfont\catcode`\^=\active\def^{\ifmmode\sp\else\^{}\fi}\catcode`\%=\active\def%{\%}$\mathdefault{300}$}}%
\end{pgfscope}%
\begin{pgfscope}%
\pgfsetbuttcap%
\pgfsetroundjoin%
\definecolor{currentfill}{rgb}{0.000000,0.000000,0.000000}%
\pgfsetfillcolor{currentfill}%
\pgfsetlinewidth{0.803000pt}%
\definecolor{currentstroke}{rgb}{0.000000,0.000000,0.000000}%
\pgfsetstrokecolor{currentstroke}%
\pgfsetdash{}{0pt}%
\pgfsys@defobject{currentmarker}{\pgfqpoint{-0.048611in}{0.000000in}}{\pgfqpoint{-0.000000in}{0.000000in}}{%
\pgfpathmoveto{\pgfqpoint{-0.000000in}{0.000000in}}%
\pgfpathlineto{\pgfqpoint{-0.048611in}{0.000000in}}%
\pgfusepath{stroke,fill}%
}%
\begin{pgfscope}%
\pgfsys@transformshift{0.618942in}{2.784780in}%
\pgfsys@useobject{currentmarker}{}%
\end{pgfscope}%
\end{pgfscope}%
\begin{pgfscope}%
\definecolor{textcolor}{rgb}{0.000000,0.000000,0.000000}%
\pgfsetstrokecolor{textcolor}%
\pgfsetfillcolor{textcolor}%
\pgftext[x=0.329012in, y=2.741377in, left, base]{\color{textcolor}{\rmfamily\fontsize{9.000000}{10.800000}\selectfont\catcode`\^=\active\def^{\ifmmode\sp\else\^{}\fi}\catcode`\%=\active\def%{\%}$\mathdefault{350}$}}%
\end{pgfscope}%
\begin{pgfscope}%
\definecolor{textcolor}{rgb}{0.000000,0.000000,0.000000}%
\pgfsetstrokecolor{textcolor}%
\pgfsetfillcolor{textcolor}%
\pgftext[x=0.273457in,y=1.676042in,,bottom,rotate=90.000000]{\color{textcolor}{\rmfamily\fontsize{10.000000}{12.000000}\selectfont\catcode`\^=\active\def^{\ifmmode\sp\else\^{}\fi}\catcode`\%=\active\def%{\%}Total de pares para revis\~ao}}%
\end{pgfscope}%
\begin{pgfscope}%
\pgfsetrectcap%
\pgfsetmiterjoin%
\pgfsetlinewidth{0.803000pt}%
\definecolor{currentstroke}{rgb}{0.000000,0.000000,0.000000}%
\pgfsetstrokecolor{currentstroke}%
\pgfsetdash{}{0pt}%
\pgfpathmoveto{\pgfqpoint{0.618942in}{0.502083in}}%
\pgfpathlineto{\pgfqpoint{0.618942in}{2.850000in}}%
\pgfusepath{stroke}%
\end{pgfscope}%
\begin{pgfscope}%
\pgfsetrectcap%
\pgfsetmiterjoin%
\pgfsetlinewidth{0.803000pt}%
\definecolor{currentstroke}{rgb}{0.000000,0.000000,0.000000}%
\pgfsetstrokecolor{currentstroke}%
\pgfsetdash{}{0pt}%
\pgfpathmoveto{\pgfqpoint{0.618942in}{0.502083in}}%
\pgfpathlineto{\pgfqpoint{4.650000in}{0.502083in}}%
\pgfusepath{stroke}%
\end{pgfscope}%
\begin{pgfscope}%
\definecolor{textcolor}{rgb}{0.000000,0.000000,0.000000}%
\pgfsetstrokecolor{textcolor}%
\pgfsetfillcolor{textcolor}%
\pgftext[x=1.138716in,y=2.543466in,,base]{\color{textcolor}{\rmfamily\fontsize{9.000000}{10.800000}\bfseries\selectfont\catcode`\^=\active\def^{\ifmmode\sp\else\^{}\fi}\catcode`\%=\active\def%{\%}308}}%
\end{pgfscope}%
\begin{pgfscope}%
\definecolor{textcolor}{rgb}{0.000000,0.000000,0.000000}%
\pgfsetstrokecolor{textcolor}%
\pgfsetfillcolor{textcolor}%
\pgftext[x=2.634471in,y=1.056453in,,base]{\color{textcolor}{\rmfamily\fontsize{9.000000}{10.800000}\bfseries\selectfont\catcode`\^=\active\def^{\ifmmode\sp\else\^{}\fi}\catcode`\%=\active\def%{\%}80}}%
\end{pgfscope}%
\begin{pgfscope}%
\definecolor{textcolor}{rgb}{0.000000,0.000000,0.000000}%
\pgfsetstrokecolor{textcolor}%
\pgfsetfillcolor{textcolor}%
\pgftext[x=4.130225in,y=1.154282in,,base]{\color{textcolor}{\rmfamily\fontsize{9.000000}{10.800000}\bfseries\selectfont\catcode`\^=\active\def^{\ifmmode\sp\else\^{}\fi}\catcode`\%=\active\def%{\%}95}}%
\end{pgfscope}%
\end{pgfpicture}%
\makeatother%
\endgroup%

\caption{Volume total de pares encaminhados para revisão manual por método.}
\label{fig:epi-revisoes}
\end{figure}

\begin{figure}[!ht]
\centering
%% Creator: Matplotlib, PGF backend
%%
%% To include the figure in your LaTeX document, write
%%   \input{<filename>.pgf}
%%
%% Make sure the required packages are loaded in your preamble
%%   \usepackage{pgf}
%%
%% Also ensure that all the required font packages are loaded; for instance,
%% the lmodern package is sometimes necessary when using math font.
%%   \usepackage{lmodern}
%%
%% Figures using additional raster images can only be included by \input if
%% they are in the same directory as the main LaTeX file. For loading figures
%% from other directories you can use the `import` package
%%   \usepackage{import}
%%
%% and then include the figures with
%%   \import{<path to file>}{<filename>.pgf}
%%
%% Matplotlib used the following preamble
%%   \def\mathdefault#1{#1}
%%   \everymath=\expandafter{\the\everymath\displaystyle}
%%   \IfFileExists{scrextend.sty}{
%%     \usepackage[fontsize=10.000000pt]{scrextend}
%%   }{
%%     \renewcommand{\normalsize}{\fontsize{10.000000}{12.000000}\selectfont}
%%     \normalsize
%%   }
%%   
%%   \makeatletter\@ifpackageloaded{underscore}{}{\usepackage[strings]{underscore}}\makeatother
%%
\begingroup%
\makeatletter%
\begin{pgfpicture}%
\pgfpathrectangle{\pgfpointorigin}{\pgfqpoint{4.800000in}{3.000000in}}%
\pgfusepath{use as bounding box, clip}%
\begin{pgfscope}%
\pgfsetbuttcap%
\pgfsetmiterjoin%
\definecolor{currentfill}{rgb}{1.000000,1.000000,1.000000}%
\pgfsetfillcolor{currentfill}%
\pgfsetlinewidth{0.000000pt}%
\definecolor{currentstroke}{rgb}{1.000000,1.000000,1.000000}%
\pgfsetstrokecolor{currentstroke}%
\pgfsetdash{}{0pt}%
\pgfpathmoveto{\pgfqpoint{0.000000in}{0.000000in}}%
\pgfpathlineto{\pgfqpoint{4.800000in}{0.000000in}}%
\pgfpathlineto{\pgfqpoint{4.800000in}{3.000000in}}%
\pgfpathlineto{\pgfqpoint{0.000000in}{3.000000in}}%
\pgfpathlineto{\pgfqpoint{0.000000in}{0.000000in}}%
\pgfpathclose%
\pgfusepath{fill}%
\end{pgfscope}%
\begin{pgfscope}%
\pgfsetbuttcap%
\pgfsetmiterjoin%
\definecolor{currentfill}{rgb}{1.000000,1.000000,1.000000}%
\pgfsetfillcolor{currentfill}%
\pgfsetlinewidth{0.000000pt}%
\definecolor{currentstroke}{rgb}{0.000000,0.000000,0.000000}%
\pgfsetstrokecolor{currentstroke}%
\pgfsetstrokeopacity{0.000000}%
\pgfsetdash{}{0pt}%
\pgfpathmoveto{\pgfqpoint{0.490470in}{0.502083in}}%
\pgfpathlineto{\pgfqpoint{4.650000in}{0.502083in}}%
\pgfpathlineto{\pgfqpoint{4.650000in}{2.850000in}}%
\pgfpathlineto{\pgfqpoint{0.490470in}{2.850000in}}%
\pgfpathlineto{\pgfqpoint{0.490470in}{0.502083in}}%
\pgfpathclose%
\pgfusepath{fill}%
\end{pgfscope}%
\begin{pgfscope}%
\pgfpathrectangle{\pgfqpoint{0.490470in}{0.502083in}}{\pgfqpoint{4.159530in}{2.347917in}}%
\pgfusepath{clip}%
\pgfsetbuttcap%
\pgfsetmiterjoin%
\definecolor{currentfill}{rgb}{1.000000,0.498039,0.054902}%
\pgfsetfillcolor{currentfill}%
\pgfsetlinewidth{0.401500pt}%
\definecolor{currentstroke}{rgb}{0.000000,0.000000,0.000000}%
\pgfsetstrokecolor{currentstroke}%
\pgfsetdash{}{0pt}%
\pgfpathmoveto{\pgfqpoint{0.679540in}{0.502083in}}%
\pgfpathlineto{\pgfqpoint{1.374081in}{0.502083in}}%
\pgfpathlineto{\pgfqpoint{1.374081in}{2.435662in}}%
\pgfpathlineto{\pgfqpoint{0.679540in}{2.435662in}}%
\pgfpathlineto{\pgfqpoint{0.679540in}{0.502083in}}%
\pgfpathclose%
\pgfusepath{stroke,fill}%
\end{pgfscope}%
\begin{pgfscope}%
\pgfpathrectangle{\pgfqpoint{0.490470in}{0.502083in}}{\pgfqpoint{4.159530in}{2.347917in}}%
\pgfusepath{clip}%
\pgfsetbuttcap%
\pgfsetmiterjoin%
\definecolor{currentfill}{rgb}{0.172549,0.627451,0.172549}%
\pgfsetfillcolor{currentfill}%
\pgfsetlinewidth{0.401500pt}%
\definecolor{currentstroke}{rgb}{0.000000,0.000000,0.000000}%
\pgfsetstrokecolor{currentstroke}%
\pgfsetdash{}{0pt}%
\pgfpathmoveto{\pgfqpoint{2.222964in}{0.502083in}}%
\pgfpathlineto{\pgfqpoint{2.917506in}{0.502083in}}%
\pgfpathlineto{\pgfqpoint{2.917506in}{0.805931in}}%
\pgfpathlineto{\pgfqpoint{2.222964in}{0.805931in}}%
\pgfpathlineto{\pgfqpoint{2.222964in}{0.502083in}}%
\pgfpathclose%
\pgfusepath{stroke,fill}%
\end{pgfscope}%
\begin{pgfscope}%
\pgfpathrectangle{\pgfqpoint{0.490470in}{0.502083in}}{\pgfqpoint{4.159530in}{2.347917in}}%
\pgfusepath{clip}%
\pgfsetbuttcap%
\pgfsetmiterjoin%
\definecolor{currentfill}{rgb}{0.090196,0.745098,0.811765}%
\pgfsetfillcolor{currentfill}%
\pgfsetlinewidth{0.401500pt}%
\definecolor{currentstroke}{rgb}{0.000000,0.000000,0.000000}%
\pgfsetstrokecolor{currentstroke}%
\pgfsetdash{}{0pt}%
\pgfpathmoveto{\pgfqpoint{3.766389in}{0.502083in}}%
\pgfpathlineto{\pgfqpoint{4.460930in}{0.502083in}}%
\pgfpathlineto{\pgfqpoint{4.460930in}{0.888799in}}%
\pgfpathlineto{\pgfqpoint{3.766389in}{0.888799in}}%
\pgfpathlineto{\pgfqpoint{3.766389in}{0.502083in}}%
\pgfpathclose%
\pgfusepath{stroke,fill}%
\end{pgfscope}%
\begin{pgfscope}%
\pgfsetbuttcap%
\pgfsetroundjoin%
\definecolor{currentfill}{rgb}{0.000000,0.000000,0.000000}%
\pgfsetfillcolor{currentfill}%
\pgfsetlinewidth{0.803000pt}%
\definecolor{currentstroke}{rgb}{0.000000,0.000000,0.000000}%
\pgfsetstrokecolor{currentstroke}%
\pgfsetdash{}{0pt}%
\pgfsys@defobject{currentmarker}{\pgfqpoint{0.000000in}{-0.048611in}}{\pgfqpoint{0.000000in}{0.000000in}}{%
\pgfpathmoveto{\pgfqpoint{0.000000in}{0.000000in}}%
\pgfpathlineto{\pgfqpoint{0.000000in}{-0.048611in}}%
\pgfusepath{stroke,fill}%
}%
\begin{pgfscope}%
\pgfsys@transformshift{1.026810in}{0.502083in}%
\pgfsys@useobject{currentmarker}{}%
\end{pgfscope}%
\end{pgfscope}%
\begin{pgfscope}%
\definecolor{textcolor}{rgb}{0.000000,0.000000,0.000000}%
\pgfsetstrokecolor{textcolor}%
\pgfsetfillcolor{textcolor}%
\pgftext[x=0.840209in, y=0.318056in, left, base]{\color{textcolor}{\rmfamily\fontsize{9.000000}{10.800000}\selectfont\catcode`\^=\active\def^{\ifmmode\sp\else\^{}\fi}\catcode`\%=\active\def%{\%}Limiar}}%
\end{pgfscope}%
\begin{pgfscope}%
\definecolor{textcolor}{rgb}{0.000000,0.000000,0.000000}%
\pgfsetstrokecolor{textcolor}%
\pgfsetfillcolor{textcolor}%
\pgftext[x=0.894770in, y=0.181250in, left, base]{\color{textcolor}{\rmfamily\fontsize{9.000000}{10.800000}\selectfont\catcode`\^=\active\def^{\ifmmode\sp\else\^{}\fi}\catcode`\%=\active\def%{\%}($\geq$8)}}%
\end{pgfscope}%
\begin{pgfscope}%
\pgfsetbuttcap%
\pgfsetroundjoin%
\definecolor{currentfill}{rgb}{0.000000,0.000000,0.000000}%
\pgfsetfillcolor{currentfill}%
\pgfsetlinewidth{0.803000pt}%
\definecolor{currentstroke}{rgb}{0.000000,0.000000,0.000000}%
\pgfsetstrokecolor{currentstroke}%
\pgfsetdash{}{0pt}%
\pgfsys@defobject{currentmarker}{\pgfqpoint{0.000000in}{-0.048611in}}{\pgfqpoint{0.000000in}{0.000000in}}{%
\pgfpathmoveto{\pgfqpoint{0.000000in}{0.000000in}}%
\pgfpathlineto{\pgfqpoint{0.000000in}{-0.048611in}}%
\pgfusepath{stroke,fill}%
}%
\begin{pgfscope}%
\pgfsys@transformshift{2.570235in}{0.502083in}%
\pgfsys@useobject{currentmarker}{}%
\end{pgfscope}%
\end{pgfscope}%
\begin{pgfscope}%
\definecolor{textcolor}{rgb}{0.000000,0.000000,0.000000}%
\pgfsetstrokecolor{textcolor}%
\pgfsetfillcolor{textcolor}%
\pgftext[x=2.330258in, y=0.318056in, left, base]{\color{textcolor}{\rmfamily\fontsize{9.000000}{10.800000}\selectfont\catcode`\^=\active\def^{\ifmmode\sp\else\^{}\fi}\catcode`\%=\active\def%{\%}ML-only}}%
\end{pgfscope}%
\begin{pgfscope}%
\definecolor{textcolor}{rgb}{0.000000,0.000000,0.000000}%
\pgfsetstrokecolor{textcolor}%
\pgfsetfillcolor{textcolor}%
\pgftext[x=2.196444in, y=0.189583in, left, base]{\color{textcolor}{\rmfamily\fontsize{9.000000}{10.800000}\selectfont\catcode`\^=\active\def^{\ifmmode\sp\else\^{}\fi}\catcode`\%=\active\def%{\%}RF+SMOTE}}%
\end{pgfscope}%
\begin{pgfscope}%
\pgfsetbuttcap%
\pgfsetroundjoin%
\definecolor{currentfill}{rgb}{0.000000,0.000000,0.000000}%
\pgfsetfillcolor{currentfill}%
\pgfsetlinewidth{0.803000pt}%
\definecolor{currentstroke}{rgb}{0.000000,0.000000,0.000000}%
\pgfsetstrokecolor{currentstroke}%
\pgfsetdash{}{0pt}%
\pgfsys@defobject{currentmarker}{\pgfqpoint{0.000000in}{-0.048611in}}{\pgfqpoint{0.000000in}{0.000000in}}{%
\pgfpathmoveto{\pgfqpoint{0.000000in}{0.000000in}}%
\pgfpathlineto{\pgfqpoint{0.000000in}{-0.048611in}}%
\pgfusepath{stroke,fill}%
}%
\begin{pgfscope}%
\pgfsys@transformshift{4.113660in}{0.502083in}%
\pgfsys@useobject{currentmarker}{}%
\end{pgfscope}%
\end{pgfscope}%
\begin{pgfscope}%
\definecolor{textcolor}{rgb}{0.000000,0.000000,0.000000}%
\pgfsetstrokecolor{textcolor}%
\pgfsetfillcolor{textcolor}%
\pgftext[x=3.798597in, y=0.318056in, left, base]{\color{textcolor}{\rmfamily\fontsize{9.000000}{10.800000}\selectfont\catcode`\^=\active\def^{\ifmmode\sp\else\^{}\fi}\catcode`\%=\active\def%{\%}Hybrid-OR}}%
\end{pgfscope}%
\begin{pgfscope}%
\definecolor{textcolor}{rgb}{0.000000,0.000000,0.000000}%
\pgfsetstrokecolor{textcolor}%
\pgfsetfillcolor{textcolor}%
\pgftext[x=3.819772in, y=0.189583in, left, base]{\color{textcolor}{\rmfamily\fontsize{9.000000}{10.800000}\selectfont\catcode`\^=\active\def^{\ifmmode\sp\else\^{}\fi}\catcode`\%=\active\def%{\%}RF+Rules}}%
\end{pgfscope}%
\begin{pgfscope}%
\pgfsetbuttcap%
\pgfsetroundjoin%
\definecolor{currentfill}{rgb}{0.000000,0.000000,0.000000}%
\pgfsetfillcolor{currentfill}%
\pgfsetlinewidth{0.803000pt}%
\definecolor{currentstroke}{rgb}{0.000000,0.000000,0.000000}%
\pgfsetstrokecolor{currentstroke}%
\pgfsetdash{}{0pt}%
\pgfsys@defobject{currentmarker}{\pgfqpoint{-0.048611in}{0.000000in}}{\pgfqpoint{-0.000000in}{0.000000in}}{%
\pgfpathmoveto{\pgfqpoint{-0.000000in}{0.000000in}}%
\pgfpathlineto{\pgfqpoint{-0.048611in}{0.000000in}}%
\pgfusepath{stroke,fill}%
}%
\begin{pgfscope}%
\pgfsys@transformshift{0.490470in}{0.502083in}%
\pgfsys@useobject{currentmarker}{}%
\end{pgfscope}%
\end{pgfscope}%
\begin{pgfscope}%
\definecolor{textcolor}{rgb}{0.000000,0.000000,0.000000}%
\pgfsetstrokecolor{textcolor}%
\pgfsetfillcolor{textcolor}%
\pgftext[x=0.329012in, y=0.458681in, left, base]{\color{textcolor}{\rmfamily\fontsize{9.000000}{10.800000}\selectfont\catcode`\^=\active\def^{\ifmmode\sp\else\^{}\fi}\catcode`\%=\active\def%{\%}$\mathdefault{0}$}}%
\end{pgfscope}%
\begin{pgfscope}%
\pgfsetbuttcap%
\pgfsetroundjoin%
\definecolor{currentfill}{rgb}{0.000000,0.000000,0.000000}%
\pgfsetfillcolor{currentfill}%
\pgfsetlinewidth{0.803000pt}%
\definecolor{currentstroke}{rgb}{0.000000,0.000000,0.000000}%
\pgfsetstrokecolor{currentstroke}%
\pgfsetdash{}{0pt}%
\pgfsys@defobject{currentmarker}{\pgfqpoint{-0.048611in}{0.000000in}}{\pgfqpoint{-0.000000in}{0.000000in}}{%
\pgfpathmoveto{\pgfqpoint{-0.000000in}{0.000000in}}%
\pgfpathlineto{\pgfqpoint{-0.048611in}{0.000000in}}%
\pgfusepath{stroke,fill}%
}%
\begin{pgfscope}%
\pgfsys@transformshift{0.490470in}{0.778309in}%
\pgfsys@useobject{currentmarker}{}%
\end{pgfscope}%
\end{pgfscope}%
\begin{pgfscope}%
\definecolor{textcolor}{rgb}{0.000000,0.000000,0.000000}%
\pgfsetstrokecolor{textcolor}%
\pgfsetfillcolor{textcolor}%
\pgftext[x=0.329012in, y=0.734906in, left, base]{\color{textcolor}{\rmfamily\fontsize{9.000000}{10.800000}\selectfont\catcode`\^=\active\def^{\ifmmode\sp\else\^{}\fi}\catcode`\%=\active\def%{\%}$\mathdefault{1}$}}%
\end{pgfscope}%
\begin{pgfscope}%
\pgfsetbuttcap%
\pgfsetroundjoin%
\definecolor{currentfill}{rgb}{0.000000,0.000000,0.000000}%
\pgfsetfillcolor{currentfill}%
\pgfsetlinewidth{0.803000pt}%
\definecolor{currentstroke}{rgb}{0.000000,0.000000,0.000000}%
\pgfsetstrokecolor{currentstroke}%
\pgfsetdash{}{0pt}%
\pgfsys@defobject{currentmarker}{\pgfqpoint{-0.048611in}{0.000000in}}{\pgfqpoint{-0.000000in}{0.000000in}}{%
\pgfpathmoveto{\pgfqpoint{-0.000000in}{0.000000in}}%
\pgfpathlineto{\pgfqpoint{-0.048611in}{0.000000in}}%
\pgfusepath{stroke,fill}%
}%
\begin{pgfscope}%
\pgfsys@transformshift{0.490470in}{1.054534in}%
\pgfsys@useobject{currentmarker}{}%
\end{pgfscope}%
\end{pgfscope}%
\begin{pgfscope}%
\definecolor{textcolor}{rgb}{0.000000,0.000000,0.000000}%
\pgfsetstrokecolor{textcolor}%
\pgfsetfillcolor{textcolor}%
\pgftext[x=0.329012in, y=1.011132in, left, base]{\color{textcolor}{\rmfamily\fontsize{9.000000}{10.800000}\selectfont\catcode`\^=\active\def^{\ifmmode\sp\else\^{}\fi}\catcode`\%=\active\def%{\%}$\mathdefault{2}$}}%
\end{pgfscope}%
\begin{pgfscope}%
\pgfsetbuttcap%
\pgfsetroundjoin%
\definecolor{currentfill}{rgb}{0.000000,0.000000,0.000000}%
\pgfsetfillcolor{currentfill}%
\pgfsetlinewidth{0.803000pt}%
\definecolor{currentstroke}{rgb}{0.000000,0.000000,0.000000}%
\pgfsetstrokecolor{currentstroke}%
\pgfsetdash{}{0pt}%
\pgfsys@defobject{currentmarker}{\pgfqpoint{-0.048611in}{0.000000in}}{\pgfqpoint{-0.000000in}{0.000000in}}{%
\pgfpathmoveto{\pgfqpoint{-0.000000in}{0.000000in}}%
\pgfpathlineto{\pgfqpoint{-0.048611in}{0.000000in}}%
\pgfusepath{stroke,fill}%
}%
\begin{pgfscope}%
\pgfsys@transformshift{0.490470in}{1.330760in}%
\pgfsys@useobject{currentmarker}{}%
\end{pgfscope}%
\end{pgfscope}%
\begin{pgfscope}%
\definecolor{textcolor}{rgb}{0.000000,0.000000,0.000000}%
\pgfsetstrokecolor{textcolor}%
\pgfsetfillcolor{textcolor}%
\pgftext[x=0.329012in, y=1.287357in, left, base]{\color{textcolor}{\rmfamily\fontsize{9.000000}{10.800000}\selectfont\catcode`\^=\active\def^{\ifmmode\sp\else\^{}\fi}\catcode`\%=\active\def%{\%}$\mathdefault{3}$}}%
\end{pgfscope}%
\begin{pgfscope}%
\pgfsetbuttcap%
\pgfsetroundjoin%
\definecolor{currentfill}{rgb}{0.000000,0.000000,0.000000}%
\pgfsetfillcolor{currentfill}%
\pgfsetlinewidth{0.803000pt}%
\definecolor{currentstroke}{rgb}{0.000000,0.000000,0.000000}%
\pgfsetstrokecolor{currentstroke}%
\pgfsetdash{}{0pt}%
\pgfsys@defobject{currentmarker}{\pgfqpoint{-0.048611in}{0.000000in}}{\pgfqpoint{-0.000000in}{0.000000in}}{%
\pgfpathmoveto{\pgfqpoint{-0.000000in}{0.000000in}}%
\pgfpathlineto{\pgfqpoint{-0.048611in}{0.000000in}}%
\pgfusepath{stroke,fill}%
}%
\begin{pgfscope}%
\pgfsys@transformshift{0.490470in}{1.606985in}%
\pgfsys@useobject{currentmarker}{}%
\end{pgfscope}%
\end{pgfscope}%
\begin{pgfscope}%
\definecolor{textcolor}{rgb}{0.000000,0.000000,0.000000}%
\pgfsetstrokecolor{textcolor}%
\pgfsetfillcolor{textcolor}%
\pgftext[x=0.329012in, y=1.563583in, left, base]{\color{textcolor}{\rmfamily\fontsize{9.000000}{10.800000}\selectfont\catcode`\^=\active\def^{\ifmmode\sp\else\^{}\fi}\catcode`\%=\active\def%{\%}$\mathdefault{4}$}}%
\end{pgfscope}%
\begin{pgfscope}%
\pgfsetbuttcap%
\pgfsetroundjoin%
\definecolor{currentfill}{rgb}{0.000000,0.000000,0.000000}%
\pgfsetfillcolor{currentfill}%
\pgfsetlinewidth{0.803000pt}%
\definecolor{currentstroke}{rgb}{0.000000,0.000000,0.000000}%
\pgfsetstrokecolor{currentstroke}%
\pgfsetdash{}{0pt}%
\pgfsys@defobject{currentmarker}{\pgfqpoint{-0.048611in}{0.000000in}}{\pgfqpoint{-0.000000in}{0.000000in}}{%
\pgfpathmoveto{\pgfqpoint{-0.000000in}{0.000000in}}%
\pgfpathlineto{\pgfqpoint{-0.048611in}{0.000000in}}%
\pgfusepath{stroke,fill}%
}%
\begin{pgfscope}%
\pgfsys@transformshift{0.490470in}{1.883211in}%
\pgfsys@useobject{currentmarker}{}%
\end{pgfscope}%
\end{pgfscope}%
\begin{pgfscope}%
\definecolor{textcolor}{rgb}{0.000000,0.000000,0.000000}%
\pgfsetstrokecolor{textcolor}%
\pgfsetfillcolor{textcolor}%
\pgftext[x=0.329012in, y=1.839808in, left, base]{\color{textcolor}{\rmfamily\fontsize{9.000000}{10.800000}\selectfont\catcode`\^=\active\def^{\ifmmode\sp\else\^{}\fi}\catcode`\%=\active\def%{\%}$\mathdefault{5}$}}%
\end{pgfscope}%
\begin{pgfscope}%
\pgfsetbuttcap%
\pgfsetroundjoin%
\definecolor{currentfill}{rgb}{0.000000,0.000000,0.000000}%
\pgfsetfillcolor{currentfill}%
\pgfsetlinewidth{0.803000pt}%
\definecolor{currentstroke}{rgb}{0.000000,0.000000,0.000000}%
\pgfsetstrokecolor{currentstroke}%
\pgfsetdash{}{0pt}%
\pgfsys@defobject{currentmarker}{\pgfqpoint{-0.048611in}{0.000000in}}{\pgfqpoint{-0.000000in}{0.000000in}}{%
\pgfpathmoveto{\pgfqpoint{-0.000000in}{0.000000in}}%
\pgfpathlineto{\pgfqpoint{-0.048611in}{0.000000in}}%
\pgfusepath{stroke,fill}%
}%
\begin{pgfscope}%
\pgfsys@transformshift{0.490470in}{2.159436in}%
\pgfsys@useobject{currentmarker}{}%
\end{pgfscope}%
\end{pgfscope}%
\begin{pgfscope}%
\definecolor{textcolor}{rgb}{0.000000,0.000000,0.000000}%
\pgfsetstrokecolor{textcolor}%
\pgfsetfillcolor{textcolor}%
\pgftext[x=0.329012in, y=2.116033in, left, base]{\color{textcolor}{\rmfamily\fontsize{9.000000}{10.800000}\selectfont\catcode`\^=\active\def^{\ifmmode\sp\else\^{}\fi}\catcode`\%=\active\def%{\%}$\mathdefault{6}$}}%
\end{pgfscope}%
\begin{pgfscope}%
\pgfsetbuttcap%
\pgfsetroundjoin%
\definecolor{currentfill}{rgb}{0.000000,0.000000,0.000000}%
\pgfsetfillcolor{currentfill}%
\pgfsetlinewidth{0.803000pt}%
\definecolor{currentstroke}{rgb}{0.000000,0.000000,0.000000}%
\pgfsetstrokecolor{currentstroke}%
\pgfsetdash{}{0pt}%
\pgfsys@defobject{currentmarker}{\pgfqpoint{-0.048611in}{0.000000in}}{\pgfqpoint{-0.000000in}{0.000000in}}{%
\pgfpathmoveto{\pgfqpoint{-0.000000in}{0.000000in}}%
\pgfpathlineto{\pgfqpoint{-0.048611in}{0.000000in}}%
\pgfusepath{stroke,fill}%
}%
\begin{pgfscope}%
\pgfsys@transformshift{0.490470in}{2.435662in}%
\pgfsys@useobject{currentmarker}{}%
\end{pgfscope}%
\end{pgfscope}%
\begin{pgfscope}%
\definecolor{textcolor}{rgb}{0.000000,0.000000,0.000000}%
\pgfsetstrokecolor{textcolor}%
\pgfsetfillcolor{textcolor}%
\pgftext[x=0.329012in, y=2.392259in, left, base]{\color{textcolor}{\rmfamily\fontsize{9.000000}{10.800000}\selectfont\catcode`\^=\active\def^{\ifmmode\sp\else\^{}\fi}\catcode`\%=\active\def%{\%}$\mathdefault{7}$}}%
\end{pgfscope}%
\begin{pgfscope}%
\pgfsetbuttcap%
\pgfsetroundjoin%
\definecolor{currentfill}{rgb}{0.000000,0.000000,0.000000}%
\pgfsetfillcolor{currentfill}%
\pgfsetlinewidth{0.803000pt}%
\definecolor{currentstroke}{rgb}{0.000000,0.000000,0.000000}%
\pgfsetstrokecolor{currentstroke}%
\pgfsetdash{}{0pt}%
\pgfsys@defobject{currentmarker}{\pgfqpoint{-0.048611in}{0.000000in}}{\pgfqpoint{-0.000000in}{0.000000in}}{%
\pgfpathmoveto{\pgfqpoint{-0.000000in}{0.000000in}}%
\pgfpathlineto{\pgfqpoint{-0.048611in}{0.000000in}}%
\pgfusepath{stroke,fill}%
}%
\begin{pgfscope}%
\pgfsys@transformshift{0.490470in}{2.711887in}%
\pgfsys@useobject{currentmarker}{}%
\end{pgfscope}%
\end{pgfscope}%
\begin{pgfscope}%
\definecolor{textcolor}{rgb}{0.000000,0.000000,0.000000}%
\pgfsetstrokecolor{textcolor}%
\pgfsetfillcolor{textcolor}%
\pgftext[x=0.329012in, y=2.668484in, left, base]{\color{textcolor}{\rmfamily\fontsize{9.000000}{10.800000}\selectfont\catcode`\^=\active\def^{\ifmmode\sp\else\^{}\fi}\catcode`\%=\active\def%{\%}$\mathdefault{8}$}}%
\end{pgfscope}%
\begin{pgfscope}%
\definecolor{textcolor}{rgb}{0.000000,0.000000,0.000000}%
\pgfsetstrokecolor{textcolor}%
\pgfsetfillcolor{textcolor}%
\pgftext[x=0.273457in,y=1.676042in,,bottom,rotate=90.000000]{\color{textcolor}{\rmfamily\fontsize{10.000000}{12.000000}\selectfont\catcode`\^=\active\def^{\ifmmode\sp\else\^{}\fi}\catcode`\%=\active\def%{\%}Revis\~oes por par verdadeiro}}%
\end{pgfscope}%
\begin{pgfscope}%
\pgfsetrectcap%
\pgfsetmiterjoin%
\pgfsetlinewidth{0.803000pt}%
\definecolor{currentstroke}{rgb}{0.000000,0.000000,0.000000}%
\pgfsetstrokecolor{currentstroke}%
\pgfsetdash{}{0pt}%
\pgfpathmoveto{\pgfqpoint{0.490470in}{0.502083in}}%
\pgfpathlineto{\pgfqpoint{0.490470in}{2.850000in}}%
\pgfusepath{stroke}%
\end{pgfscope}%
\begin{pgfscope}%
\pgfsetrectcap%
\pgfsetmiterjoin%
\pgfsetlinewidth{0.803000pt}%
\definecolor{currentstroke}{rgb}{0.000000,0.000000,0.000000}%
\pgfsetstrokecolor{currentstroke}%
\pgfsetdash{}{0pt}%
\pgfpathmoveto{\pgfqpoint{0.490470in}{0.502083in}}%
\pgfpathlineto{\pgfqpoint{4.650000in}{0.502083in}}%
\pgfusepath{stroke}%
\end{pgfscope}%
\begin{pgfscope}%
\definecolor{textcolor}{rgb}{0.000000,0.000000,0.000000}%
\pgfsetstrokecolor{textcolor}%
\pgfsetfillcolor{textcolor}%
\pgftext[x=1.026810in,y=2.468809in,,base]{\color{textcolor}{\rmfamily\fontsize{9.000000}{10.800000}\bfseries\selectfont\catcode`\^=\active\def^{\ifmmode\sp\else\^{}\fi}\catcode`\%=\active\def%{\%}7.0}}%
\end{pgfscope}%
\begin{pgfscope}%
\definecolor{textcolor}{rgb}{0.000000,0.000000,0.000000}%
\pgfsetstrokecolor{textcolor}%
\pgfsetfillcolor{textcolor}%
\pgftext[x=2.570235in,y=0.839078in,,base]{\color{textcolor}{\rmfamily\fontsize{9.000000}{10.800000}\bfseries\selectfont\catcode`\^=\active\def^{\ifmmode\sp\else\^{}\fi}\catcode`\%=\active\def%{\%}1.1}}%
\end{pgfscope}%
\begin{pgfscope}%
\definecolor{textcolor}{rgb}{0.000000,0.000000,0.000000}%
\pgfsetstrokecolor{textcolor}%
\pgfsetfillcolor{textcolor}%
\pgftext[x=4.113660in,y=0.921946in,,base]{\color{textcolor}{\rmfamily\fontsize{9.000000}{10.800000}\bfseries\selectfont\catcode`\^=\active\def^{\ifmmode\sp\else\^{}\fi}\catcode`\%=\active\def%{\%}1.4}}%
\end{pgfscope}%
\end{pgfpicture}%
\makeatother%
\endgroup%

\caption{Custo operacional: número de revisões manuais por par verdadeiro recuperado.}
\label{fig:epi-custo}
\end{figure}

\section{Dois \textit{pipelines} e escolha de ponto operacional}
\label{sec:cap6-dois-pipelines}

A exploração sistemática de 828 configurações na fronteira de Pareto (Tabela \ref{tab:pareto-frontier}, Figura \ref{fig:pareto-frontier}) demonstrou que precisão e sensibilidade não podem ser maximizadas simultaneamente, resultado esperado pela teoria de decisão estatística, mas aqui quantificado para o contexto específico do \textit{linkage} SIM$\times$Sinan. A implicação prática é direta: não existe um único ponto operacional ótimo, e a escolha deve ser orientada pela finalidade do \textit{linkage} \cite{Harron2017linkagequality}.

O \textit{pipeline} orientado à vigilância prioriza sensibilidade. A configuração RF+SMOTE (limiar 0,5) atingiu F$_1$-Score de 0,916$\pm$0,026 na validação cruzada estratificada com cinco partições, apresentando o menor coeficiente de variação entre as configurações avaliadas (Tabela \ref{tab:cv-5fold}). Essa estabilidade é relevante para uso em rotina: um classificador cuja sensibilidade oscila entre partições comprometeria a comparabilidade temporal dos indicadores de mortalidade. Em contextos de monitoramento contínuo e análise de tendências, o custo de um falso negativo (óbito não detectado) supera o custo de um falso positivo (par incorreto encaminhado para revisão), justificando a operação em ponto de maior sensibilidade \cite{Bartholomay2014improved}.

No polo oposto, o \textit{pipeline} de confirmação prioriza precisão. Configurações híbridas do tipo AND (por exemplo, Gradient Boosting com limiar $\geq 0,6$ combinado a regras com escore $\geq 5$) exigem concordância simultânea entre o classificador probabilístico e evidências determinísticas em atributos-chave. O estudo de ablação (Tabela \ref{tab:ablation-best-category}) mostrou que essa exigência dupla eleva a precisão ao custo de sensibilidade moderada, padrão compatível com ações administrativas, relatórios formais de encerramento e situações em que a auditoria posterior é inviável ou onerosa.

A análise de robustez ao desbalanceamento fortalece a confiança nos dois \textit{pipelines}. O F$_1$-Score permaneceu na faixa de 0,880 a 0,918 sob nove estratégias de reamostragem distintas (Tabela \ref{tab:imbalance-sensitivity}), indicando que o desempenho não depende criticamente de uma única técnica de balanceamento. Combinações do tipo OR, por sua vez, melhoraram a estabilidade média do F$_1$-Score sob validação cruzada ao recuperar pares verdadeiros por duas vias complementares (probabilística ou determinística), reduzindo a dependência de uma única fonte de evidência \cite{Doidge2018demystifying}.

A escolha entre os \textit{pipelines}, portanto, não é meramente técnica: envolve análise de risco, volume esperado de revisão e finalidade institucional do \textit{linkage}. O \textit{framework} configurável proposto nesta tese formaliza essa decisão, tornando explícitos os compromissos envolvidos e permitindo que gestores e epidemiologistas selecionem o ponto operacional mais adequado ao seu contexto, com base em evidência empírica documentada.

% =============================================================================
% RESERVA PARA CAPÍTULO 7 (DISCUSSÃO)
% Seções de pensamento sistêmico, episódios de cuidado e painéis BI,
% reescritas para conectar-se aos resultados empíricos da tese.
% =============================================================================


% -----------------------------------------------------------------------------
% §7.3 - Pensamento sistêmico e crises sanitárias
% Conectado aos resultados: recuperação de óbitos, zona cinzenta, COVID-19
% -----------------------------------------------------------------------------
\section[Pensamento sistêmico e crises sanitárias]{Pensamento sistêmico e o impacto de crises sanitárias}\label{sec:pensamento-sistemico}

Sistemas de saúde constituem sistemas complexos adaptativos, compostos por múltiplos agentes que interagem de forma não linear e produzem dinâmicas emergentes irredutíveis à análise isolada de seus componentes \cite{Plsek2001complexity}. O pensamento sistêmico propõe que a compreensão de fenômenos de saúde exige a consideração de interrelações entre elementos, de rotas de retroalimentação (\textit{feedback loops}) e de efeitos não intencionais de intervenções, em contraposição à lógica reducionista que isola variáveis e relações causais lineares \cite{Sterman2000business, Luke2012systems}.

Os resultados desta tese oferecem evidência concreta dessa interdependência. A recuperação de 24 óbitos adicionais pelo classificador RF+SMOTE, em relação ao limiar ingênuo $\geq 8$, ilustra como uma falha localizada na cadeia de informação (a classificação imprecisa de potenciais pares na zona cinzenta) propaga seus efeitos para indicadores de nível populacional: taxas de mortalidade subestimadas, encerramentos de fichas de notificação incorretos e, consequentemente, alocação de recursos baseada em informação incompleta \cite{Bartholomay2014improved, Lima2020tbquality}. Sob a ótica sistêmica, o pós-processamento por aprendizado de máquina atua como mecanismo de correção de um ponto de fragilidade na rota de retroalimentação entre o registro do óbito e a vigilância epidemiológica.

A aplicação do pensamento sistêmico à saúde pública tem ganhado relevância na análise de problemas que envolvem múltiplos determinantes sociais, ambientais e organizacionais \cite{DiezRoux2011complexity, Adam2012systems}. No campo das doenças infecciosas, essa perspectiva permite reconhecer que o desfecho do tratamento de um paciente com tuberculose não depende exclusivamente da eficácia do esquema terapêutico, mas de uma rede de fatores que inclui o acesso oportuno ao diagnóstico, a organização dos serviços de atenção primária, a disponibilidade de exames laboratoriais e a capacidade de articulação entre os pontos da rede \cite{Mendes2011redes}. A concentração de aproximadamente 47\% dos pares verdadeiros na zona cinzenta (escores 5 a 8) reflete, em parte, essa complexidade: registros com preenchimento incompleto, variações na grafia de nomes ou inconsistências em datas de nascimento são manifestações, no nível dos dados, de fragilidades organizacionais nos pontos de registro do sistema de saúde.

Crises sanitárias recentes amplificam essas fragilidades de forma documentada. A pandemia de COVID-19 provocou sobrecarga nos serviços hospitalares e de atenção primária no Brasil, com redução no número de notificações de tuberculose, interrupção de tratamentos e aumento de desfechos desfavoráveis \cite{Ranzani2021covid, Maia2022covid_tb}. A queda na detecção de casos de TB durante a pandemia não reflete necessariamente redução na incidência da doença, mas retração do acesso aos serviços de diagnóstico e desarticulação de rotinas de vigilância \cite{Hallal2020covid}. Nesse cenário, a degradação adicional da qualidade dos registros (campos incompletos, atrasos na digitação, acúmulo de fichas não encerradas) tende a aumentar a proporção de pares candidatos que caem na zona cinzenta, tornando o pós-processamento supervisionado ainda mais necessário. A robustez do classificador demonstrada pela análise de sensibilidade ao desbalanceamento (F$_1$-Score entre 0,880 e 0,918 sob nove estratégias) sugere que o \textit{framework} proposto pode manter desempenho estável mesmo em períodos de deterioração da qualidade dos dados, embora essa hipótese requeira validação em coortes pandêmicas.

A contribuição do \textit{framework} configurável, nesse contexto, transcende o ganho preditivo: ao tornar explícitos os compromissos entre precisão e sensibilidade em cada ponto operacional da fronteira de Pareto (828 configurações exploradas), o sistema permite que gestores ajustem a sensibilidade do monitoramento conforme o cenário epidemiológico vigente. Em períodos de crise, a operação em ponto de maior sensibilidade pode compensar parcialmente a perda de notificações, funcionando como mecanismo de alerta para desorganizações sistêmicas detectáveis pelo aumento de vínculos recuperados na zona cinzenta.


% -----------------------------------------------------------------------------
% §7.4 - Episódios de cuidado e itinerário terapêutico
% Conectado aos resultados: SHAP, NOMEMAE, zona cinzenta, framework
% -----------------------------------------------------------------------------
\section{Episódios de cuidado e itinerário terapêutico}\label{sec:episodios-itinerario}

O conceito de episódio de cuidado, introduzido por Hornbrook, Hurtado e Johnson \citeyearpar{Hornbrook1985episodes}, designa o conjunto articulado de serviços de saúde prestados a um indivíduo em relação a um problema clínico específico, ao longo de um período temporal definido. Diferentemente da análise de eventos isolados (uma internação, uma consulta, um exame), a abordagem por episódios reconhece que o cuidado em saúde constitui processo longitudinal, no qual as interações do paciente com diferentes pontos do sistema assistencial são interdependentes.

A reconstrução de episódios de cuidado a partir de dados administrativos depende, fundamentalmente, da capacidade de identificar registros referentes a um mesmo indivíduo em diferentes bases de informação. Na ausência de um identificador unívoco no SUS, essa tarefa exige o emprego de técnicas de \textit{linkage} que possibilitem vincular, por exemplo, a notificação de um caso de TB no Sinan à eventual declaração de óbito no SIM \cite{Coeli2021suboptimal}. A acurácia dessa vinculação determina a completude do episódio reconstruído: cada par verdadeiro não identificado representa uma lacuna no itinerário terapêutico do paciente.

Os achados de interpretabilidade desta tese lançam luz sobre a dinâmica dessa reconstrução. A análise SHAP (\textit{SHapley Additive exPlanations}) revelou que, na zona cinzenta do escore agregado, o nome da mãe (NOMEMAE) emerge como atributo dominante nas decisões do classificador (Tabela \ref{tab:shap-importance}, Figura \ref{fig:shap-summary}) \cite{Lundberg2017shap, Lundberg2020treeshap}. Essa predominância possui interpretação epidemiológica relevante: quando a evidência de nome próprio e data de nascimento se torna ambígua (por erros de digitação, homônimos ou variações de grafia), o nome da mãe funciona como âncora de identificação familiar que preserva a vinculação mesmo diante de inconsistências nos demais campos. Tal padrão sugere que o itinerário terapêutico do paciente com TB, quando reconstruído por \textit{linkage}, depende criticamente da qualidade de preenchimento de campos que, na prática assistencial, são frequentemente tratados como secundários.

Essa constatação dialoga com a literatura sobre itinerário terapêutico na perspectiva antropológica. O itinerário, conceito que designa o percurso empreendido pelo indivíduo na busca por cuidado, englobando serviços formais, estratégias informais e barreiras de acesso \cite{Cabral2008itinerario, Gerhardt2006itinerario}, pode ser operacionalizado como sequência temporal de eventos registrados em diferentes sistemas quando o \textit{linkage} é bem-sucedido. Os 24 óbitos adicionais recuperados pelo RF+SMOTE representam, cada um, a restauração de um elo entre a trajetória de cuidado (registrada no Sinan) e o desfecho final (registrado no SIM). Sem essa vinculação, o episódio de cuidado permanece incompleto: o sistema de vigilância registra uma notificação sem encerramento definitivo, e o óbito permanece dissociado de sua história clínica, comprometendo tanto o cálculo de indicadores quanto a avaliação da qualidade da assistência prestada \cite{Bartholomay2020drtb}.

A transparência proporcionada pela análise SHAP amplia a utilidade do \textit{framework} para a gestão da informação em saúde \cite{Markus2021role}. Ao identificar que o nome da mãe é o atributo decisivo na zona de incerteza, o sistema fornece aos gestores uma evidência objetiva para direcionar ações de melhoria da qualidade do preenchimento: investir na completude do campo de nome da mãe nos formulários de notificação e nas declarações de óbito pode reduzir a proporção de pares que caem na zona cinzenta, diminuindo a dependência do pós-processamento supervisionado e fortalecendo a capacidade de reconstrução automática dos episódios de cuidado.


% -----------------------------------------------------------------------------
% §7.5 - Painéis de monitoramento e inteligência de dados em saúde
% Conectado aos resultados: framework configurável, pipelines, Pareto
% -----------------------------------------------------------------------------
\section{Painéis de monitoramento e inteligência de dados em saúde}\label{sec:paineis-bi}

A crescente disponibilidade de dados em saúde tem motivado o desenvolvimento de painéis de monitoramento (\textit{dashboards}) e sistemas de inteligência de dados (\textit{Business Intelligence}, BI) voltados à gestão e à vigilância em saúde \cite{Kimball2013dw}. Essas ferramentas permitem a agregação, a visualização e a análise de indicadores em tempo oportuno, subsidiando a tomada de decisão em diferentes níveis do sistema. Entretanto, parte expressiva dessas iniciativas limita-se à apresentação descritiva de dados provenientes de bases isoladas, sem incorporar a integração de fontes necessária para a construção de indicadores de processo e resultado.

Os \textit{pipelines} configuráveis propostos nesta tese podem alimentar painéis de monitoramento com dados vinculados de maior qualidade, expandindo o repertório de indicadores disponíveis para a gestão. No \textit{pipeline} de vigilância (RF+SMOTE, limiar 0,5), a saída classificada forneceria, em fluxo operacional, a lista atualizada de óbitos vinculados a notificações de TB, permitindo a construção de indicadores longitudinais: proporção de óbitos entre casos notificados, tempo entre notificação e óbito, e taxa de encerramento oportuno. A operação desse \textit{pipeline} em ponto de alta sensibilidade (F$_1$-Score = 0,916 $\pm$ 0,026 na validação cruzada) asseguraria cobertura ampla dos eventos, condição necessária para que o painel reflita a realidade epidemiológica com mínima subestimação \cite{Bartholomay2014improved}.

Já no \textit{pipeline} de confirmação (combinação híbrida AND), as classificações de alta confiança integrariam módulos de investigação individual nos painéis, apresentando ao epidemiologista apenas os pares de alta confiança que dispensam revisão adicional. A fronteira de Pareto, com 828 configurações exploradas, oferece ao gestor um mapa de possibilidades operacionais que pode ser incorporado à interface do painel como ferramenta de ajuste dinâmico: em períodos de sobrecarga (por exemplo, durante uma crise sanitária), o operador poderia deslocar o ponto operacional em direção a maior sensibilidade; em períodos de rotina, retornaria ao ponto de maior precisão, reduzindo o volume de alertas \cite{Harron2017linkagequality}.

A interpretabilidade do classificador via SHAP agrega uma dimensão adicional aos painéis. A exibição das contribuições dos atributos para cada decisão de classificação (em particular, a predominância do nome da mãe na zona cinzenta) permitiria que o painel funcionasse não apenas como ferramenta de monitoramento epidemiológico, mas também como instrumento de gestão da qualidade da informação: regiões ou unidades de saúde com alta proporção de pares classificados na zona cinzenta poderiam ser priorizadas para ações de capacitação em preenchimento de registros \cite{Lundberg2020treeshap, Markus2021role}. Essa integração entre a saída do \textit{framework} de \textit{linkage} e painéis de BI representa, assim, a materialização do ciclo completo de inteligência epidemiológica: do registro no ponto de atenção, passando pela vinculação probabilística e classificação supervisionada, até a geração de indicadores acionáveis para a tomada de decisão em saúde pública.


\section{Contribuição do \textit{framework} em relação ao uso isolado de classificadores}
\label{sec:cap6-contribuicao-framework}

Cabe explicitar a distinção entre o \textit{framework} proposto neste trabalho e a aplicação isolada de um classificador, como a Floresta Aleatória (\textit{Random Forest}). A execução direta de um único modelo com hiperparâmetros padrão produz um ponto de operação no espaço precisão-sensibilidade, sem oferecer ao operador elementos para avaliar alternativas ou ajustar o comportamento do sistema ao contexto de uso. O \textit{framework} desenvolvido nesta tese distingue-se por cinco componentes integrados.

Primeiro, o Comparador de Registros \cite{Jardim2024comparador, Lucena2013algoritmos} fornece 29 subescores de similaridade campo a campo, e não apenas o escore final agregado (\textit{nota final}), ampliando substancialmente a representação de cada par candidato e possibilitando que os classificadores explorem padrões de concordância parcial entre campos distintos. Essa granularidade favorece a detecção de pares com concordância heterogênea entre campos, situação frequente na zona cinzenta do \textit{linkage}.

Em seguida, a etapa de engenharia de atributos (\textit{feature engineering}) deriva sistematicamente variáveis adicionais a partir dos escores brutos, incluindo termos de interação entre campos, escores quadráticos e indicadores binários de concordância com limiares diferenciados por estratégia, conforme detalhado na Seção~\ref{sec:engenharia-atributos}. Essa camada de transformação enriquece o espaço de atributos e permite a captura de evidências combinadas que não seriam acessíveis a um classificador operando diretamente sobre o escore agregado.

Adicionalmente, o estudo de ablação sistemático, abrangendo mais de 70 configurações experimentais (variações de classificador, estratégia de balanceamento, combinação híbrida e limiar de decisão), substitui a escolha \textit{ad hoc} de um modelo por uma exploração exaustiva do espaço de configurações. Essa exploração viabiliza a identificação de configurações dominantes e a construção da fronteira de Pareto (\textit{Pareto frontier}) no plano precisão $\times$ sensibilidade.

Como quarto diferencial, a fronteira de Pareto assim obtida não constitui artefato analítico isolado, mas instrumento de apoio à decisão: permite ao gestor ou pesquisador selecionar explicitamente o compromisso entre falsos positivos e falsos negativos adequado ao objetivo do estudo, seja vigilância epidemiológica (prioridade à sensibilidade), seja construção de coortes analíticas de alta confiabilidade (prioridade à precisão). A execução isolada de uma Floresta Aleatória com parâmetros padrão produz um único ponto nesse espaço; o \textit{framework} fornece a fronteira completa e o protocolo operacional para selecionar o ponto adequado.

Por fim, os dois \textit{pipelines} pré-configurados (vigilância e confirmação de alta confiança), descritos na Seção~\ref{sec:cap6-dois-pipelines}, traduzem os achados experimentais em recomendações operacionais diretamente aplicáveis, reduzindo a dependência de expertise em aprendizado de máquina por parte dos profissionais de saúde que operam o \textit{linkage}. O resultado é um protocolo que ultrapassa a mera aplicação de um classificador. Em conjunto, esses componentes configuram abordagem de pós-processamento reprodutível, configurável e auditável para a qualificação de dados vinculados em saúde. A extensão desse \textit{framework} para a governança formal da incerteza na zona cinzenta, com calibração por âncoras, política de custo assimétrica e revisão assistida por modelo de linguagem, é apresentada no Capítulo~\ref{cap:gzcmd}. O arcabouço resultante reduz em três ordens de grandeza o volume de pares encaminhados à revisão humana, concentrando o esforço do revisor nos casos genuinamente ambíguos.

\section{Limitações e generalização}
\label{sec:cap6-limitacoes}

A principal limitação deste estudo reside na construção do padrão-ouro. O conjunto de referência foi elaborado por um único revisor do IESC-UFRJ por meio de revisão clerical, busca manual complementar e classificação de cada candidato como par ou não par. Embora a revisão por avaliador único seja frequente em estudos operacionais de \textit{linkage} \cite{Gupta2022framework, Gupta2024manual}, a ausência de um segundo revisor independente impede tanto o cálculo de concordância inter-avaliadores (\textit{kappa} de Cohen) quanto a resolução de casos ambíguos por consenso, o que pode introduzir viés individual na rotulagem, especialmente na zona cinzenta, onde a ambiguidade dos registros é maior \cite{Harron2017linkagequality}. Estudos futuros que incorporem dupla revisão independente e métricas de concordância poderão estimar a magnitude dessa incerteza.

Uma segunda limitação refere-se ao risco de falsos negativos no padrão-ouro decorrentes de falhas na etapa de blocagem (\textit{blocking}). Se um par verdadeiro não foi gerado como candidato pelo OpenRecLink em nenhum dos passos de blocagem, ele estará ausente do universo avaliado, e tanto o classificador quanto a revisão manual serão incapazes de recuperá-lo. Essa limitação é inerente a qualquer estudo de \textit{linkage} probabilístico baseado em blocagem \cite{Doidge2019linkageerror} e implica que as métricas de sensibilidade reportadas nesta tese representam estimativas condicionais ao conjunto de candidatos gerados, não estimativas absolutas da capacidade de detecção. Abordagens alternativas, como a geração de identidades sintéticas \cite{Lam2024synthetic} ou a validação cruzada com múltiplas estratégias de blocagem, poderiam mitigar parcialmente essa restrição.

O desbalanceamento extremo (1:249) entre pares verdadeiros e não pares constitui desafio estatístico relevante, pois pequenas variações na taxa de falsos positivos podem gerar grandes volumes de revisão. A análise de sensibilidade com nove estratégias de reamostragem indicou estabilidade do F$_1$-Score (0,880 a 0,918), sugerindo robustez do classificador a escolhas de balanceamento. Entretanto, a transposição direta dos limiares ótimos para bases com proporções de desbalanceamento distintas requer recalibração, uma vez que os valores preditivos positivo e negativo dependem da prevalência \cite{Shaw2022biases}.

A disponibilidade restrita de variáveis clínicas no conjunto exportado pelo comparador de registros limita análises epidemiológicas mais detalhadas. Atributos como sexo, raça/cor, bairro de residência e comorbidades, embora presentes nas bases originais do SIM e do Sinan, não integraram o vetor de características por razões de escopo e privacidade, impedindo a estratificação do desempenho do classificador por subgrupos populacionais. A literatura documenta que a qualidade do preenchimento varia por região e por sistema de informação \cite{Lima2020tbquality, Bartholomay2020drtb}, de modo que a generalização para outras localidades e pares de bases (por exemplo, SIM$\times$SIH ou Sinan$\times$GAL) depende de revalidação externa com padrões-ouro locais \cite{Coeli2021suboptimal}.

A validação de arquiteturas de aprendizado profundo \textit{(deep learning)} em bases de maior volume configura agenda relevante para trabalhos futuros. A escolha por algoritmos baseados em árvore fundamenta-se na dimensão reduzida do conjunto positivo (247 pares verdadeiros) e na estrutura tabular dos atributos, condições que podem ser superadas em bases administrativas nacionais, as quais acumulam milhões de registros anuais \cite{Shaw2022biases}. Redes neurais profundas, particularmente arquiteturas híbridas que combinem camadas convolucionais para processamento de campos textuais com camadas densas para atributos estruturados, poderiam capturar padrões latentes em dados de linkage em larga escala, embora a vantagem preditiva dessas arquiteturas sobre métodos tradicionais em bases epidemiológicas brasileiras permaneça por quantificar.

Apesar dessas restrições, os resultados sustentam que o pós-processamento supervisionado com \textit{framework} configurável constitui avanço operacional em relação a limiares fixos e regras ad hoc. A documentação explícita dos compromissos entre precisão, sensibilidade e custo de revisão, associada à interpretabilidade via SHAP, confere transparência ao processo decisório, condição necessária para a adoção em rotinas de vigilância \cite{Markus2021role}. Parte das limitações aqui discutidas, em particular a dependência de limiares fixos e a escalabilidade da revisão manual, é endereçada pelo arcabouço GZ-CMD, apresentado no Capítulo~\ref{cap:gzcmd}. Esse arcabouço propõe política de decisão por perda esperada e revisão assistida por modelo de linguagem como alternativas operacionais, reduzindo de 21.620 para 41 os pares que demandam revisão humana no cenário de vigilância.

\end{chapter}
